\documentclass{uc3mpracticas}



%%%%%%%%%%%%%%%%%%%%%%%%%%%%%%%%%%%%%%%%%%%%%%%%%%%%%%%%%%%%%%%%%%%%%%%%%%%%%%%%
%%%                   Plantilla Prácticas UC3M                               %%%
%%%                Universidad Carlos III de Madrid                          %%%
%%%                   Alejandro Valverde Mahou                               %%%
%%%%%%%%%%%%%%%%%%%%%%%%%%%%%%%%%%%%%%%%%%%%%%%%%%%%%%%%%%%%%%%%%%%%%%%%%%%%%%%%

%Permitir cabeceras y pie de páginas personalizados
\pagestyle{fancy}

%Path por defecto de las imágenes
\graphicspath{ {./images/} }

%Declarar formato de encabezado y pie de página de las páginas del documento
\fancypagestyle{doc}{
  %Cabecera
  \headerpr[1]{Fase de Diseño}{\textbf{}}{Diseño de Sistemas Interactivos}
  %Pie de Página
  \footerpr{\textbf{Universidad Carlos III de Madrid}}{}{{\thepage} de \pageref{LastPage}}
}

%Declarar formato de encabezado y pie del título e indice
\fancypagestyle{titu}{%
  %Cabecera
  \headerpr{}{}{}
  %Pie de Página
  \footerpr{}{}{}
}


\appto\frontmatter{\pagestyle{titu}}
\appto\mainmatter{\pagestyle{doc}}


\begin{document}
  %Comienzo formato título
  \frontmatter


  \centeredtitle{Images/logo_help.png}{Grado en Ingeniería Informática}{Curso 2020}{Diseño de Sistemas Interactivos}{Fase de Diseño: H3lp Me}

    \vspace{2mm}

    \begin{center}
      \line(1, 0){450}
    \end{center}

    \authorsright{Alejandro Parrado Rivas}{100383453}{Adrián Sanz Gómez}{100383473}{Alejandro Valverde Mahou}{100383383}{Andrés Vinagre Blanco}{100383414}

    \newpage


    %Índice
    \tableofcontents

  \newpage

  %Comienzo formato documento general
  \mainmatter

  \section{Metodología y Documentación del Proceso de Diseño}

  Estamos diseñando para usuarios de entre 16 y 55 años, que son usuarios que tienen presente en su día a día el uso de dispositivos móviles, Internert, aplicaciones.. etc. Sobre todo, los de mayor edad
  están viendo las utilidades que les proporciona internet y cada vez se muestran más abiertos a estar conectados.

  H3lp Me es una aplicación que tiene como objetivo solventar problemas cotidianos que surgen en el día a día de las personas, generando oportunidades de empleo a tiempo parcial para los que se ofrecen a realizar trabajos propuestos por otros usuarios. De tal forma que se establece una comunicación directa entre el solicitante y el trabajador para así poder resolver las tareas punutales propuestas por el solicitante de forma rápida y concisa.
  La motivación de esta aplicación ,principalmente, es resolver de manera rápida actividades de la vida cotidiana como, por ejemplo, que no funcione un ordenador y que alguien que sepa se ofrezca a repararlo a cambio de una compensación económica o  que alguien necesite en un momento puntual que otra perona cuide de sus hijos o mascotas.



  La situación o escenario que estamos diseñando es
En cambio, las tareas y actividades que realizan los usuarios potenciales son múltiples. Muchos de ellos consultan el periódico en su smartphone a través de es, pìden comida a domicilio, realizan compras en Amazon, Aliexpress, etc., utilizan redes sociales como Facebook, Instagram, Twitter y/o aplicaciones de mensajería instantánea como, por ejemplo, WhatsApp, etc. En definitiva, usan el smartphone para facilitar su día a día tanto como para trabajo, ocio, informarse de noticias, escuchar música, reproducir vídeos y entretener a sus posibles hijos con aplicaciones para niños.

  Estamos diseñando para una situación en la que la gran mayoría de las personas usa Internet, tanto en ordenadores como en smartphones para consultar el periódico, pedir comida a domicilio, realizar compras en Amazon, Aliexpress, etc., utilizan redes sociales como Facebook, Instagram, Twitter y/o aplicaciones de mensajería instantánea como, por ejemplo, WhatsApp, Facebook Messenger etc. En definitiva, usan Internet para facilitar su día a día tanto como para trabajo, ocio, informarse de noticias, escuchar música, reproducir vídeos, etc. Además, casi todas las personas cuentan en sus casas con estos dispositivos y conexión a Internet. Con el auge de los Internet en los últimos años, los dispositivos móviles y el internet de las cosas, la mayoría de las personas de la sociedad están diariamente en contacto con la red. También, las personas de entre 35 y 50 años, han notado el cambio de lo que supone estar conectado a internet y de las ventajas del mismo, de hecho puede observarse que gente de esta edad utilizan wearables como pulseras inteligentes, smartwatches, servicios en internet como Netflix, Amazon, Spotify, Apple Music... .


  Problemas o fric points....





  \section{Fase Divergente: Conceptos de Diseño y Sketches}

  Los prototipos han sido diseñados siguiendo las heurísticas de Nielsen y los patrones de diseño. Para así conseguir un diseño web que mejore la experiencia y usabilidad de los usuarios.
  En todos los prototipos mostrados a continuación se han considerado tres pantallas o wireframes básicos que tendrá la aplicación web. Estas son la pantalla de inicio, la principal y la de perfil.

  \subsection{Prototipo 1. Andrés Vinagre Blanco.}


  \imgcenter{Images/DSIinicio.png}

  Al abrir la página web aparece una breve descripción de la plataforma además de un logo con una frase que describe el funcionamiento, especialmente dirigida a los usuarios que no estén registrados; los cuales podrán registrarse también en esta misma página inicial. Los usuarios que ya se registraron, podrán iniciar sesión (botón superior parte derecha). Por lo tanto, se aplica el patrón de diseño H2-Sign-in/New Account. Asimismo, con el logo de la marca en el centro de la página se consigue aplicar el patrón E1-Site branding, dando credibilidad y confianza a los usuarios sobre la web. Además, se aplica el patrón de diseño C1-HomePage Portal puesto que el 95% de los usuarios se fijarán en la parte izquierda y verán el botón de “Regístrate”.



  \imgcenter{Images/general.png}

  Esta es la página tras haber iniciado sesión,  donde el usuario verá el header junto un buscador que permite realizar una búsqueda de servicios, aplicándose así el patrón de diseño J1- Search Action Module. En esta página, se mantiene el logo en el centro (patrón E1-Site-Branding). En la parte derecha del header se muestra una rueda de ajustes en la que al clickar aparecería un menú y se tiene en cuenta el uso de la metáfora para el diseño. A la derecha de este botón aparece una foto de perfil que sirve para acceder a la siguiente imagen que se muestra, que es el perfil del usuario .  Asimismo, el diseño que se sigue es bastante minimalista e intuitivo siguiéndose la heurística de Nielsen 9: estética y diseño minimalista donde de menos información es mejor, puesto que no se sobrecarga al usuario. Al usar metáforas como la rueda de ajustes, la “casa” para recargar la página, el icono de tres puntos se sigue la heurística de Nielsen 2: coincidencia entre el sistema y el mundo real.
  Se tiene además un diseño basado en tarjetas donde en cada una de ellas aparecerá la información de cada servicio que se ofrece o solicita. Justo encima de esta tarjetas aparece un botón para añadir un nuevo servicio por parte del usuario. Esto implica que se tenga en cuenta las heurísticas K4-Action-Buttons (botones para hacer acciones) y K5-High-Visibility-Action-Buttons ya que el botón de acción está remarcado.

  \imgcenter{Images/perfil.png}

  En la parte izquierda  hay un menú con el que el usuario realizará las diferentes acciones. Aquí se emplea el patrón de diseño B2-Browsable Content, puesto que la información se mantiene localizada y el C1-Homepage Portal ya que las opciones de menú están a la izquierda, viéndolas un 95% de los usuarios están en esta página. Se ve de neuvo que en este menú aparece los botones de accion remarcados siguiendo las patrones K4 y K5 mencionados anteriormente. Por otro lado, en la parte derecha aparece una tarjeta donde se muestra la información del perfil y se ha tenido en cuenta incorporar un nivel y monedas para aplicar gamificación.
  En el menú el usuario puede acceder alas mejores categorias, a los servicios que ha solicitado y/o ofrecido, a una sección donde ver los amigos que tiene (H3lpers) y, por último, a la sección de chat, donde aparecerían las conversaciones que ha tenido el usuario con el resto. No obstante, estas pantallas se han considerado en una fase de diseño posterior, una vez que se tiene una idea y un concepto del diseño de la web.



  \subsection{Prototipo 2}

  \imgcenter{Images/prototipo1.png}
  \imgcenter{Images/prototipo2.png}

  \subsection{Prototipo 3}

  \imgcenter{Images/ALEJANpage_11.png}
  \imgcenter{Images/ALEJANpage_2.png}

  \subsection{Prototipo 4}

  \imgcenter{Images/home.png}
  \imgcenter{Images/search.png}
  \imgcenter{Images/profile.png}


  \section{Fase Convergente: Selección de un Concepto de Diseño}

  \section{Diseño Final: Escenario y Detalles de Diseño}








\end{document}
