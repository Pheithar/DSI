\documentclass{uc3mpracticas}



%%%%%%%%%%%%%%%%%%%%%%%%%%%%%%%%%%%%%%%%%%%%%%%%%%%%%%%%%%%%%%%%%%%%%%%%%%%%%%%%
%%%                   Plantilla Prácticas UC3M                               %%%
%%%                Universidad Carlos III de Madrid                          %%%
%%%                   Alejandro Valverde Mahou                               %%%
%%%%%%%%%%%%%%%%%%%%%%%%%%%%%%%%%%%%%%%%%%%%%%%%%%%%%%%%%%%%%%%%%%%%%%%%%%%%%%%%

%Permitir cabeceras y pie de páginas personalizados
\pagestyle{fancy}

%Path por defecto de las imágenes
\graphicspath{ {./images/} }

%Declarar formato de encabezado y pie de página de las páginas del documento
\fancypagestyle{doc}{
  %Cabecera
  \headerpr[1]{Fase de Diseño}{\textbf{}}{Diseño de Sistemas Interactivos}
  %Pie de Página
  \footerpr{\textbf{Universidad Carlos III de Madrid}}{}{{\thepage} de \pageref{LastPage}}
}

%Declarar formato de encabezado y pie del título e indice
\fancypagestyle{titu}{%
  %Cabecera
  \headerpr{}{}{}
  %Pie de Página
  \footerpr{}{}{}
}


\appto\frontmatter{\pagestyle{titu}}
\appto\mainmatter{\pagestyle{doc}}


\begin{document}
  %Comienzo formato título
  \frontmatter


  \centeredtitle{Images/logo_help.png}{Grado en Ingeniería Informática}{Curso 2020}{Diseño de Sistemas Interactivos}{Fase de Diseño: H3lp Me}

    \vspace{2mm}

    \begin{center}
      \line(1, 0){450}
    \end{center}

    \authorsright{Alejandro Parrado Arribas}{100383453}{Adrián Sanz Gómez}{100383473}{Alejandro Valverde Mahou}{100383383}{Andrés Vinagre Blanco}{100383414}

    \newpage


    %Índice
    \tableofcontents

  \newpage

  %Comienzo formato documento general
  \mainmatter

  \section{Metodología y Documentación del Proceso de Diseño}

  Estamos diseñando para usuarios de entre 16 y 55 años, que son usuarios que tienen presente en su día a día el uso de dispositivos móviles, Internet, aplicaciones, etc. Además, los de mayor edad están viendo las utilidades que les proporciona internet y cada vez se muestran más dispuestos a estar conectados y usar las neuvas tecnologías.

  \vspace{4mm}

  \textit{H3lp Me} es una aplicación que tiene como objetivo solventar problemas cotidianos que surgen en el día a día de las personas, generando oportunidades de empleo de tiempo parcial para los que se ofrecen a realizar trabajos para el resto de usuarios. De tal forma que se establece una comunicación directa entre el solicitante y el trabajador para así poder resolver las tareas puntales propuestas por el solicitante de forma rápida y concisa, evitando todo ipo de papeleos, buracracia e intemediarios.

  \vspace{2mm}

  La motivación de esta aplicación, principalmente, es resolver de manera rápida actividades de la vida cotidiana como, por ejemplo, que no funcione un ordenador y que alguien que sepa se ofrezca a repararlo a cambio de una compensación económica o que alguien necesite en un momento puntual que otra persona cuide de sus hijos o mascotas.

  \vspace{4mm}

  Las tareas y actividades que realizan los usuarios potenciales a través de Internet son múltiples. Muchos de ellos consultan el periódico en su \textit{smartphone}, piden comida a domicilio, realizan compras online, utilizan redes sociales y aplicaciones de mensajería instantánea, etc. En definitiva, usan el smartphone para facilitar su día a día, usando los dispositivos electrónicos para trabajo, ver las noticias, escuchar música, reproducir vídeos y entretenerse.

  \vspace{2mm}

  A pesar de existir a día de hoy aplicaciones y páginas web de todo tipo, no existen páginas que cumplan con el propósito de la aplicación, o al menos no existe una página popularizada que cumpla con este propósito. No es una necesiadad esencial para la sociedad, pero si puede ayudar en muchos momentos a las personas que requieran de una ayuda puntual. Además, podría ayudar a la creación de empleo, o un dinero extra para los usuarios que ofrecen los servicios.

  \vspace{4mm}

  La situación o escenario que estamos diseñando es una aplicación donde existen usuarios denominados \textit{h3lpers} que pueden ofrecer diferentes servicios, que puede solicitar cualquiera de los usuarios de la página. Esta solicitud podrá ser aceptada o rechazada por el \textit{h3lper}.


  \section{Fase Divergente: Conceptos de Diseño y Sketches}

  Los prototipos han sido diseñados siguiendo las heurísticas de Nielsen y los patrones de diseño. Para así conseguir un diseño web que mejore la experiencia y usabilidad de los usuarios.
  En todos los prototipos mostrados a continuación se han considerado tres pantallas o wireframes básicos que tendrá la aplicación web. Estas son la pantalla de inicio, la principal y la de perfil.

  \subsection{Prototipo 1}


  \imgcenter[130]{Images/DSIinicio.png}

  Al abrir la página web aparece una breve descripción de la plataforma además de un logo con una frase que describe el funcionamiento, especialmente dirigida a los usuarios que no estén registrados, que podrán registrarse también desde esta misma página inicial. Los usuarios que ya esteb registrados, podrán iniciar sesión (botón superior parte derecha). Por lo tanto, se aplica el patrón de diseño \textit{H2-Sign-in/New Account}. Asimismo, con el logo de la marca en el centro de la página se consigue aplicar el patrón \textit{E1-Site} branding, dando credibilidad y confianza a los usuarios sobre la web. Además, se aplica el patrón de diseño \textit{C1-HomePage Portal} puesto que el 95\% de los usuarios se fijarán en la parte izquierda y verán el botón de “Regístrate”.

  \vspace{6mm}

  \imgcenter[130]{Images/general.png}

  Esta es la página que se muestra tras haber iniciado sesión, donde el usuario verá la cabecera, junto un buscador que permite realizar una búsqueda filtrada de servicios, aplicándose así el patrón de diseño \textit{J1-Search Action Module}. En esta página, se mantiene el logo en el centro (patrón \textit{E1-Site-Branding}). En la parte derecha de la cabecera se muestra una rueda de ajustes en la que al pulsar aparece un menú y se tiene en cuenta el uso de la metáfora para el diseño. A la derecha de este botón aparece una foto de perfil que sirve para acceder a la siguiente imagen que se muestra, que es el perfil del usuario. El diseño que se sigue es bastante minimalista e intuitivo siguiéndose la heurística de Nielsen \textit{9: estética y diseño minimalista} donde menos información es mejor, puesto que no se sobrecarga al usuario. Al usar metáforas como la rueda de ajustes, la “casa” para recargar la página, o el icono de tres puntos se sigue la heurística de Nielsen \textit{2: coincidencia entre el sistema y el mundo real}.

  \vspace{2mm}

  Se tiene además un diseño basado en tarjetas donde en cada una de ellas aparecerá la información de cada servicio que se ofrece o solicita. Justo encima de esta tarjetas aparece un botón para añadir un nuevo servicio por parte del usuario. Esto implica que se tenga en cuenta las heurísticas \textit{K4-Action-Buttons} (botones para hacer acciones) y \textit{K5-High-Visibility-Action-Buttons} ya que el botón de acción está remarcado.

  \vspace{6mm}

  \imgcenter[130]{Images/perfil.png}

  En la parte izquierda  hay un menú con el que el usuario realizará las diferentes acciones. Aquí se emplea el patrón de diseño \textit{B2-Browsable Content}, puesto que la información se mantiene localizada y el \textit{C1-Homepage Portal} ya que las opciones de menú están a la izquierda, viéndolas un 95\% de los usuarios están en esta página. Se ve de nuevo que en este menú aparece los botones de accion remarcados siguiendo las patrones K4 y K5 mencionados anteriormente. Por otro lado, en la parte derecha aparece una tarjeta donde se muestra la información del perfil y se ha tenido en cuenta incorporar un nivel y monedas para aplicar gamificación.

  \vspace{2mm}

  En el menú, el usuario puede acceder a las mejores categorias, a los servicios que ha solicitado y/o ofrecido, a una sección donde ver los amigos que tiene y, por último, a la sección de chat, donde aparecerían las conversaciones que ha tenido el usuario con el resto. No obstante, estas pantallas se han considerado en una fase de diseño posterior, una vez que se tenga una idea y un concepto del diseño de la web.



  \subsection{Prototipo 2}

  \imgcenter{Images/prototipo1.png}
  \imgcenter{Images/prototipo2.png}

  \subsection{Prototipo 3}

  Este primer diseño pertenece a la página principal, que a su vez aparece una búsqueda de servicios que se ofrecen. Está un poco basada a la aplicación Wallapop, previamente analizada en el estado del arte. Cuenta con un diseño minimalista, fondo blanco, reduciendo al máximo el número de elementos para evitar su saturación y sea agradable a la vista del usuario.

  \imgcenter[130]{Images/ALEJANpage_11.png}

  En la barra de arriba (header); a mano izquierda, se encuentra el logo de la aplicación, siempre que se pulse te mandara a la pagina inicial. En el centro hay una barra de búsqueda que permite le permite al usuario realizar una búsqueda por las palabras claves que introduzca previamente por teclado. A la derecha se encuentra la foto de perfil de los usuarios, esta zona tiene dos estados ya que depende si hay un usuario logueado o no. En el caso de que no hubiera nadie logueado aparecerían dos botones; “Registrarse” y “Iniciar Sesión”, que permitirían registrarse a la aplicación nuevos usuarios o conectarse usuarios ya registrados respectivamente. En el caso de que una persona estuviera conectada a la aplicación, aparecería su foto de perfil; tal y como se muestra en el diseño, y justo a su izquierda una flechita que al pulsarla desplegaria un PopUp con varias opciones (Ver perfil, Chat, Anuncios, Ajustes, Cerrar Sesión y Soporte).
  Esta estructura del header se mantiene durante todas las distintas vistas de la app.
  Mas abajo del header se encuentra el cuerpo de la aplicación, como se comentó anteriormente, se muestran anuncios con una estructura de tarjetas. En los anuncios se muestra una foto que ocupa el 75% de la tarjeta y que sube el usuario a la hora de publicar los anuncios. Mas abajo aparece el servicio que se ofrece y a la derecha de este aparecen dos botones, un corazón y un bocadillo; sirven para guardar el anuncio y poder consultar más tarde en tu perfil e iniciar un chat con esa persona respectivamente.

  \imgcenter[130]{Images/ALEJANpage_2.png}

  Este segundo diseño pertenece a la vista de un usuario a un perfil. Se pueden diferenciar dos partes en el diseño, la primera pertenece a la tarjeta de la izquierda donde se puede ver información general del usuario. La zona superior se encuentra una estrella que simboliza que el usuario esta suscrito a una membresía premium dentro de la aplicación, en el centro hay una foto del usuario y a la derecha su nivel. Hacia abajo se encuentra el nombre de la persona, la ubicación donde trabaja el usuario que puede ser elegida por los usuarios, un botón de contacto para iniciar un chat y un botón de seguir para poder guardar el perfil.

  \subsection{Prototipo 4}

  En este prototipo la cabecera es común para todas las páginas. Esta cabecera incluye una imagen del logo de la página, para que el usuario tenga en todo momento conocimiento de en que página se encuentra, y crear un sentimiento de marca, tal y como indica el patrón \textit{E1-Site-Branding}.

  \vspace{4mm}

  \imgcenter[130]{Images/home.png}

  La página principal explica de forma breve el propósito de la página, para que así los nuevos usuarios sean capaces de decidir si están interesados o no. Además, muestra la principal funcionalidad de la página, que es buscar servicios, con un buscador, y unas indicaciones de como usarlo. También tiene una imágen del logo de la página, pero en mayor tamaño, para reforzar la imágen de marca. Esta página de inicio se basa en los patrones \textit{J1-Search Action Module}, \textit{C1-Homepage Portal} y \textit{B2-Browsable Content}, además del \textit{E1-Site-Branding} ya mencionado.

  \vspace{8mm}

  \imgcenter[130]{Images/search.png}

  Una vez el usuario realiza una búsqueda, los servicios que correspondan con esa búsqueda se mostraran en pantalla. Si el usuario está interesado en alguno de ellos, podrá pulsar "ver más" para abrir la página del servicio y solicitarlo si así lo desea.

  \vspace{2mm}

  Esta página esta diseñada segun tarjetas individuales, para que sea sencillo identificar cada uno de los servicios. y siguiendo un estilo sencillo, limpio y minimalista en la medida de lo posible, que permita de un solo vistazo ver la información del servicio.

  \vspace{2mm}

  Esta ventana cumple con los patrones \textit{J1-Search Action Module} y \textit{K5-High-Visibility-Action-Buttons} principalmente, además de seguir la heurística \textit{9: estética y diseño minimalista} de Nielsen.

  \vspace{8mm}

  \imgcenter[130]{Images/profile.png}

  Por último el perfil muestra la información de cada uno de los usuarios, con tanto los servicios ofrecidos como los solicitados. También mustra el nombre, localización y nivel del usuario, y una pequeña señal en forma de estrella que indica si el usuario es \textit{h3lper} o no.

  \vspace{2mm}

  Se sigue la heuristica \textit{9: estética y diseño minimalista} de Nielsen, la \textit{6: reconocimiento antes que recuerdo}, y la \textit{4: consistencia y estandarización}, pues los anuncios siguen la misma forma, y todas las páginas mantienen la misma estética.


  \section{Fase Convergente: Selección de un Concepto de Diseño}

  \section{Diseño Final: Escenario y Detalles de Diseño}








\end{document}
