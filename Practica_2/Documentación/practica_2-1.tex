\documentclass{uc3mpracticas}



%%%%%%%%%%%%%%%%%%%%%%%%%%%%%%%%%%%%%%%%%%%%%%%%%%%%%%%%%%%%%%%%%%%%%%%%%%%%%%%%
%%%                   Plantilla Prácticas UC3M                               %%%
%%%                Universidad Carlos III de Madrid                          %%%
%%%                   Alejandro Valverde Mahou                               %%%
%%%%%%%%%%%%%%%%%%%%%%%%%%%%%%%%%%%%%%%%%%%%%%%%%%%%%%%%%%%%%%%%%%%%%%%%%%%%%%%%

%Permitir cabeceras y pie de páginas personalizados
\pagestyle{fancy}

%Path por defecto de las imágenes
\graphicspath{ {./images/} }

%Declarar formato de encabezado y pie de página de las páginas del documento
\fancypagestyle{doc}{
  %Cabecera
  \headerpr[1]{Fase de Diseño}{\textbf{}}{Diseño de Sistemas Interactivos}
  %Pie de Página
  \footerpr{\textbf{Universidad Carlos III de Madrid}}{}{{\thepage} de \pageref{LastPage}}
}

%Declarar formato de encabezado y pie del título e indice
\fancypagestyle{titu}{%
  %Cabecera
  \headerpr{}{}{}
  %Pie de Página
  \footerpr{}{}{}
}


\appto\frontmatter{\pagestyle{titu}}
\appto\mainmatter{\pagestyle{doc}}


\begin{document}
  %Comienzo formato título
  \frontmatter


  \centeredtitle{Images/logo_help.png}{Grado en Ingeniería Informática}{Curso 2020}{Diseño de Sistemas Interactivos}{Fase de Diseño: H3lp Me}

    \vspace{2mm}

    \begin{center}
      \line(1, 0){450}
    \end{center}

    \authorsright{Alejandro Parrado Rivas}{100383453}{Adrián Sanz Gómez}{100383473}{Alejandro Valverde Mahou}{100383383}{Andrés Vinagre Blanco}{100383414}

    \newpage


    %Índice
    \tableofcontents

  \newpage

  %Comienzo formato documento general
  \mainmatter

  \section{Metodología y Documentación del Proceso de Diseño}

  Estamos diseñando para usuarios de entre 16 y 55 años, que son usuarios que tienen presente en su día a día el uso de dispositivos móviles, Internert, aplicaciones.. etc. Sobre todo, los de mayor edad
  están viendo las utilidades que les proporciona internet y cada vez se muestran más abiertos a estar conectados.

  H3lp Me es una aplicación que tiene como objetivo solventar problemas cotidianos que surgen en el día a día de las personas, generando oportunidades de empleo a tiempo parcial para los que se ofrecen a realizar trabajos propuestos por otros usuarios. De tal forma que se establece una comunicación directa entre el solicitante y el trabajador para así poder resolver las tareas punutales propuestas por el solicitante de forma rápida y concisa.
  La motivación de esta aplicación ,principalmente, es resolver de manera rápida actividades de la vida cotidiana como, por ejemplo, que no funcione un ordenador y que alguien que sepa se ofrezca a repararlo a cambio de una compensación económica o  que alguien necesite en un momento puntual que otra perona cuide de sus hijos o mascotas.





  La situación o escenario que estamos diseñando....

  P


  \subsection{Prototipo 1}

  \imgcenter{Images/DSIinicio.png}
  \imgcenter{Images/general.png}
  \imgcenter{Images/perfil.png}

  \subsection{Prototipo 2}

  \imgcenter{Images/prototipo1.png}
  \imgcenter{Images/prototipo2.png}

  \subsection{Prototipo 3}

  \imgcenter{Images/ALEJANpage_11.png}
  \imgcenter{Images/ALEJANpage_2.png}

  \subsection{Prototipo 4}

  \imgcenter{Images/home.png}
  \imgcenter{Images/search.png}
  \imgcenter{Images/profile.png}


  \section{Fase Divergente: Conceptos de Diseño y Sketches}

  \section{Fase Convergente: Selección de un Concepto de Diseño}

  \section{Diseño Final: Escenario y Detalles de Diseño}








\end{document}
