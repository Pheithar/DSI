\documentclass{uc3mpracticas}



%%%%%%%%%%%%%%%%%%%%%%%%%%%%%%%%%%%%%%%%%%%%%%%%%%%%%%%%%%%%%%%%%%%%%%%%%%%%%%%%
%%%                   Plantilla Prácticas UC3M                               %%%
%%%                Universidad Carlos III de Madrid                          %%%
%%%                   Alejandro Valverde Mahou                               %%%
%%%%%%%%%%%%%%%%%%%%%%%%%%%%%%%%%%%%%%%%%%%%%%%%%%%%%%%%%%%%%%%%%%%%%%%%%%%%%%%%

%Permitir cabeceras y pie de páginas personalizados
\pagestyle{fancy}

%Path por defecto de las imágenes
\graphicspath{ {./images/} }

%Declarar formato de encabezado y pie de página de las páginas del documento
\fancypagestyle{doc}{
  %Cabecera
  \headerpr[1]{Fase de Diseño}{\textbf{}}{Diseño de Sistemas Interactivos}
  %Pie de Página
  \footerpr{\textbf{Universidad Carlos III de Madrid}}{}{{\thepage} de \pageref{LastPage}}
}

%Declarar formato de encabezado y pie del título e indice
\fancypagestyle{titu}{%
  %Cabecera
  \headerpr{}{}{}
  %Pie de Página
  \footerpr{}{}{}
}


\appto\frontmatter{\pagestyle{titu}}
\appto\mainmatter{\pagestyle{doc}}


\begin{document}
  %Comienzo formato título
  \frontmatter


  \centeredtitle{Images/logo_help.png}{Grado en Ingeniería Informática}{Curso 2020}{Diseño de Sistemas Interactivos}{Fase de Diseño: H3lp Me}

    \vspace{2mm}

    \begin{center}
      \line(1, 0){450}
    \end{center}

    \authorsright{Alejandro Parrado Arribas}{100383453}{Adrián Sanz Gómez}{100383473}{Alejandro Valverde Mahou}{100383383}{Andrés Vinagre Blanco}{100383414}

    \newpage


    %Índice
    \tableofcontents

  \newpage

  %Comienzo formato documento general
  \mainmatter

  \section{Metodología y Documentación del Proceso de Diseño}

    \textit{H3lp Me} es una aplicación que tiene como objetivo solventar problemas cotidianos que surgen en el día a día de las personas, generando oportunidades de empleo de tiempo parcial para los que se ofrecen a realizar trabajos para el resto de usuarios. De tal forma que se establece una comunicación directa entre el solicitante y el trabajador para así poder resolver las tareas puntales propuestas por el solicitante de forma rápida y concisa, evitando todo ipo de papeleos, buracracia e intemediarios.

  \vspace{4mm}


  La motivación de esta aplicación, principalmente, es resolver de manera rápida actividades de la vida cotidiana como, por ejemplo, que no funcione un ordenador y que alguien que sepa se ofrezca a repararlo a cambio de una compensación económica o que alguien necesite en un momento puntual que otra persona cuide de sus hijos o mascotas.

  \vspace{2mm}

  Estamos diseñando para usuarios de entre 16 y 55 años, que son usuarios que tienen presente en su día a día el uso de dispositivos móviles, Internet, aplicaciones, etc. Además, los de mayor edad están viendo las utilidades que les proporciona internet y cada vez se muestran más dispuestos a estar conectados y usar las neuvas tecnologías.


  \vspace{4mm}

  Las tareas y actividades que realizan los usuarios potenciales a través de Internet son múltiples. Muchos de ellos consultan el periódico en su \textit{smartphone}, piden comida a domicilio, realizan compras online, utilizan redes sociales y aplicaciones de mensajería instantánea, etc. En definitiva, usan el smartphone para facilitar su día a día, usando los dispositivos electrónicos para trabajo, ver las noticias, escuchar música, reproducir vídeos y entretenerse.

  \vspace{2mm}

  A pesar de existir a día de hoy aplicaciones y páginas web de todo tipo, no existen páginas que cumplan con el propósito de la aplicación, o al menos no existe una página popularizada que cumpla con este propósito. No es una necesiadad esencial para la sociedad, pero si puede ayudar en muchos momentos a las personas que requieran de una ayuda puntual. Además, podría ayudar a la creación de empleo, o un dinero extra para los usuarios que ofrecen los servicios.

  \vspace{4mm}

  La situación o escenario que estamos diseñando es una aplicación donde existen usuarios denominados \textit{h3lpers} que pueden ofrecer diferentes servicios, que puede solicitar cualquiera de los usuarios de la página. Esta solicitud podrá ser aceptada o rechazada por el \textit{h3lper}.


  \section{Fase Divergente: Conceptos de Diseño y Sketches}

  Los prototipos han sido diseñados siguiendo las heurísticas de Nielsen y los patrones de diseño. Para así conseguir un diseño web que mejore la experiencia y usabilidad de los usuarios.
  En todos los prototipos mostrados a continuación se han considerado tres pantallas o wireframes básicos que tendrá la aplicación web. Estas son la pantalla de inicio, la principal y la de perfil.

  \subsection{Prototipo 1}


  \imgcenter[130]{Images/DSIinicio.png}

  Al abrir la página web aparece una breve descripción de la plataforma además de un logo con una frase que describe el funcionamiento, especialmente dirigida a los usuarios que no estén registrados, que podrán registrarse también desde esta misma página inicial. Los usuarios que ya estén registrados, podrán iniciar sesión (botón superior parte derecha). Por lo tanto, se aplica el patrón de diseño \textit{H2-Sign-in/New Account}. Asimismo, con el logo de la marca en el centro de la página se consigue aplicar el patrón \textit{E1-Site} branding, dando credibilidad y confianza a los usuarios sobre la web. Además, se aplica el patrón de diseño \textit{C1-HomePage Portal} puesto que el 95\% de los usuarios se fijarán en la parte izquierda y verán el botón de “Regístrate”.

  \vspace{6mm}

  \imgcenter[130]{Images/general.png}

  Esta es la página que se muestra tras haber iniciado sesión, donde el usuario verá la cabecera, junto un buscador que permite realizar una búsqueda filtrada de servicios, aplicándose así el patrón de diseño \textit{J1-Search Action Module}. En esta página, se mantiene el logo en el centro (patrón \textit{E1-Site-Branding}). En la parte derecha de la cabecera se muestra una rueda de ajustes en la que al pulsar aparece un menú y se tiene en cuenta el uso de la metáfora para el diseño. A la derecha de este botón aparece una foto de perfil que sirve para acceder a la siguiente imagen que se muestra, que es el perfil del usuario. El diseño que se sigue es bastante minimalista e intuitivo siguiéndose la heurística de Nielsen \textit{9: estética y diseño minimalista} donde menos información es mejor, puesto que no se sobrecarga al usuario. Al usar metáforas como la rueda de ajustes, la “casa” para recargar la página, o el icono de tres puntos se sigue la heurística de Nielsen \textit{2: coincidencia entre el sistema y el mundo real}.

  \vspace{2mm}

  Se tiene además un diseño basado en tarjetas donde en cada una de ellas aparecerá la información de cada servicio que se ofrece o solicita. Justo encima de esta tarjetas aparece un botón para añadir un nuevo servicio por parte del usuario. Esto implica que se tenga en cuenta las heurísticas \textit{K4-Action-Buttons} (botones para hacer acciones) y \textit{K5-High-Visibility-Action-Buttons} ya que el botón de acción está remarcado.

  \vspace{6mm}

  \imgcenter[130]{Images/perfil.png}

  En la parte izquierda  hay un menú con el que el usuario realizará las diferentes acciones. Aquí se emplea el patrón de diseño \textit{B2-Browsable Content}, puesto que la información se mantiene localizada y el \textit{C1-Homepage Portal} ya que las opciones de menú están a la izquierda, viéndolas un 95\% de los usuarios están en esta página. Se ve de nuevo que en este menú aparece los botones de accion remarcados siguiendo las patrones K4 y K5 mencionados anteriormente. Por otro lado, en la parte derecha aparece una tarjeta donde se muestra la información del perfil y se ha tenido en cuenta incorporar un nivel y monedas para aplicar gamificación.

  \vspace{2mm}

  En el menú, el usuario puede acceder a las mejores categorías, a los servicios que ha solicitado y/o ofrecido, a una sección donde ver los amigos que tiene y, por último, a la sección de chat, donde aparecerían las conversaciones que ha tenido el usuario con el resto. No obstante, estas pantallas se han considerado en una fase de diseño posterior, una vez que se tenga una idea y un concepto del diseño de la web.



  \subsection{Prototipo 2}

  \imgcenter{Images/prototipo1.png}

  Este diseño está separado en tres secciones diferenciadas: la cabecera, donde podemos encontrar los botones de registrarse e iniciar sesión y el logo, junto al nombre de la app. En el sector del medio encontramos el cuerpo, compuesto de un video introductorio de la aplicación, un apartado para registrarse y un test para recomendar servicios que puede ofrecer un usuario.

  \vspace{2mm}

  Por último, el pie de página se compone de los apartados para contactar con los administradores, un botón que te conduce a información sobre la app y otro botón sobre privacidad, además del aviso de los derechos de autor que posee la página (copyright).

  \imgcenter{Images/prototipo2.png}

  En la segunda imagen del prototipo se  aprecia la página una vez iniciada la sesión. La cabecera cambia en la parte derecha para pasar a mostrar un botón para cerrar sesión y un botón de acceso al perfil y ajustes.

  \vspace{2mm}

  En el cuerpo de la página se encuentra un apartado para los servicios ofrecidos por el usuario con un botón para poder añadir más. En cada uno de estos servicios se ve una imagen representativa y un título, además de un botón para poder ver más información sobre el servicio.

  \vspace{2mm}

  A su vez en el cuerpo también se representa un apartado donde se puede ver una lista de los amigos que tiene el usuario, así como la posibilidad de agregar otro mediante un botón.

  \vspace{3mm}

  También se incluye el test de habilidades que se encuentra en la página inicial, pero con menor tamaño. El pie de página sigue siendo el mismo.

  \subsection{Prototipo 3}

  Este primer diseño pertenece a la página principal, en la que aparece una búsqueda de servicios que ofrecidos. Está basada en la aplicación Wallapop, previamente analizada en el estado del arte. Cuenta con un diseño minimalista, un fondo blanco, intentando reducir al máximo el número de elementos para evitar su saturación y que sea agradable a la vista del usuario.

  \imgcenter[130]{Images/ALEJANpage_11.png}

  En la barra de arriba (\textit{header}), a la izquierda, se encuentra el logo de la aplicación. Siempre que se pulse se redirigirá a la pagina inicial. En el centro hay una barra de búsqueda que permite al usuario realizar una búsqueda por palabras claves introducidas por teclado. A la derecha se encuentra la foto de perfil de los usuarios, esta zona tiene dos estados ya que depende si hay un usuario con sesión iniciada o no. En el caso de que el usuario no haya iniciado sesión, aparecerán dos botones: “Registrarse” e “Iniciar Sesión”, que permitirían registrarse a la aplicación nuevos usuarios o conectarse usuarios ya registrados respectivamente. En el caso de que hubiera un usuario con sesión iniciada en la aplicación, aparecería su foto de perfil, tal y como se muestra en el diseño, y justo a su izquierda una flecha que al ser pulsada desplegaria un \textit{pop-up} con varias opciones (Ver perfil, Chat, Anuncios, Ajustes, Cerrar Sesión y Soporte).

  \vspace{2mm}

  Esta estructura del \textit{header} se mantiene durante todas las distintas vistas de la aplicación.

  \vspace{2mm}

  Debajo del \textit{header} se encuentra el cuerpo de la aplicación, como se comentó anteriormente, se muestran anuncios con una estructura de tarjetas. En los anuncios se muestra una imagen que ocupa el 75\% de la tarjeta y que sube el usuario a la hora de publicar el anuncio. Más abajo aparece el servicio que se ofrece y a la derecha de este aparecen dos botones, un corazón y un bocadillo, que sirven para guardar el anuncio y poder consultar más tarde en tu perfil e iniciar un chat con el usuario que ofrece el anuncio, respectivamente.

  \imgcenter[130]{Images/ALEJANpage_2.png}

  Este segundo diseño pertenece a la vista del perfil de un usuario. Se pueden diferenciar dos partes en el diseño, la primera pertenece a la tarjeta de la izquierda donde se puede ver información general del usuario. La zona superior se encuentra una estrella que simboliza que el usuario es un \textit{h3lper}, en el centro hay una foto del usuario y a la derecha su nivel. Debajo se encuentra el nombre de la persona, la ubicación donde trabaja el usuario que puede ser elegida por los usuarios, un botón de contacto para iniciar un chat y un botón de seguir para poder guardar el perfil.

  \subsection{Prototipo 4}

  En este prototipo la cabecera es común para todas las páginas. Esta cabecera incluye una imagen del logo de la página, para que el usuario tenga en todo momento conocimiento de en que página se encuentra, y crear un sentimiento de marca, tal y como indica el patrón \textit{E1-Site-Branding}.

  \vspace{4mm}

  \imgcenter[130]{Images/home.png}

  La página principal explica de forma breve el propósito de la página, para que así los nuevos usuarios sean capaces de decidir si están interesados o no. Además, muestra la principal funcionalidad de la página, que es buscar servicios, con un buscador, y unas indicaciones de como usarlo. También tiene una imagen del logo de la página, pero en mayor tamaño, para reforzar la imágen de marca. Esta página de inicio se basa en los patrones \textit{J1-Search Action Module}, \textit{C1-Homepage Portal} y \textit{B2-Browsable Content}, además del \textit{E1-Site-Branding} ya mencionado.

  \vspace{8mm}

  \imgcenter[130]{Images/search.png}

  Una vez el usuario realiza una búsqueda, los servicios que correspondan con esa búsqueda se mostraran en pantalla. Si el usuario está interesado en alguno de ellos, podrá pulsar "ver más" para abrir la página del servicio y solicitarlo si así lo desea.

  \vspace{2mm}

  Esta página está diseñada segun tarjetas individuales, para que sea sencillo identificar cada uno de los servicios. y siguiendo un estilo sencillo, limpio y minimalista en la medida de lo posible, que permita de un solo vistazo ver la información del servicio.

  \vspace{2mm}

  Esta ventana cumple con los patrones \textit{J1-Search Action Module} y \textit{K5-High-Visibility-Action-Buttons} principalmente, además de seguir la heurística \textit{9: estética y diseño minimalista} de Nielsen.

  \vspace{8mm}

  \imgcenter[130]{Images/profile.png}

  Por último el perfil muestra la información de cada uno de los usuarios, con tanto los servicios ofrecidos como los solicitados. También mustra el nombre, localización y nivel del usuario, y una pequeña señal en forma de estrella que indica si el usuario es \textit{h3lper} o no.

  \vspace{2mm}

  Se sigue la heuristica \textit{9: estética y diseño minimalista} de Nielsen, la \textit{6: reconocimiento antes que recuerdo}, y la \textit{4: consistencia y estandarización}, pues los anuncios siguen la misma forma, y todas las páginas mantienen la misma estética.


  \section{Fase Convergente: Selección de un Concepto de Diseño}

  En esta fase analizaremos los diseños anteriormente expuestos de la fase divergente y como se ha acoplado formando el diseño que finalmente hemos utilizado.

  \subsection{Prototipo 1}

  \subsubsection{Pros}

  Dentro de este prototipo se destaca su \textbf{minimalismo y estética}. Se puede ver un estilo orientado a interfaces propias de \textit{Apple} con colores sencillos y vistosos.

  Tiene una \textbf{estructuración} muy intuitiva y una interfaz simple pero efectiva.

  \textbf{El menú del perfil} es muy cómodo y aporta las funciones necesarias para la administración del mismo así como las funciones que implementa la app.

  \subsubsection{Contras}

  \textbf{Los degradados en los colores} quitan a la página seriedad y estilo uniforme aunque la combinación de escala de azules y blancos se implementará finalmente de otro modo.

  Tiene un \textbf{diseño} demasiado simple, aunque efectivo, sigue necesitando más opciones para el usuario así como información disponible de la app.

  A su vez, \textbf{la página principal} no aporta ninguna información además del registro, lo cual no apela al interés del usuario ni intenta convencerlo para que se una a la app.

  No posee \textbf{footer}.

  \subsection{Prototipo 2}

  \subsubsection{Pros}

  \textbf{La página principal} incluye información que apela al interés del usuario por la app. Tiene un diseño simple pero eficaz, que incorpora una estructura basada en imágenes que se centran en el registro del usuario.

  Incorpora un \textbf{footer} con los elementos principales a tener en cuenta en una app de estas características.

  La idea de \textbf{introducir un test} o un enlace a un test para medir tus habilidades y destacar el campo en el que proporcionar servicios es buena.

  \subsubsection{Contras}

  \textbf{El color de fondo} azul dificulta la navegación por la página y resulta molesto y antiestético con el resto del diseño de la página.

  \textbf{El diseño} una vez iniciado sesión es poco intuitivo y muy simple. A pesar de que incorpora la ventana de amigos y servicios, no existe ningún menú de navegación por los mismos aparte de la barra de búsqueda.

  \textbf{El contraste de la gama de azules} resulta saturadora para la vista.

  \subsection{Prototipo 3}

  \subsubsection{Pros}

  \textbf{El sistema de tarjetas} para mostrar los servicios es intuitivo y la interfaz que presenta similar a la de redes sociales actuales puede ser útil para atraer a un público familiarizado con estas.

  \textbf{La intefaz del perfil de usuario} también es muy buena, mostrando la información esencial y el método de contacto con el usuario, haciéndolo fácil de usar.

  \subsubsection{Contras}

  \textbf{El diseño} es demasiado simple y anticuado, similar en estética a sistemas antiguos como Windows XP.

  No posee \textbf{footer} y el \textbf{header} no muestra toda la información necesaria para navegar correctamente por la app ni en que estado de la misma te encuentras.

  \subsection{Prototipo 4}

  \subsubsection{Pros}

  \textbf{La cabecera} tiene la información necesaria para el usuario  posee una subcabecera con la barra de búsqueda, lo que es muy util para la navegación por la página.

  \textbf{La página principal} tiene un diseño simple y bien explicado, donde le permite al usuario buscar los servicios sin registrarse.

  \textbf{El sistema de tarjetas} para los servicios incluye la información esencial del servicio y la posibilidad de abrir una nueva ventana para ver toda la información.

  \textbf{La interfaz del perfil} es intuitiva y con un diseño fácil de entender.


  \subsubsection{Contras}

  El color de fondo azul de fondo de la página principal resalta más que el de la cabecera y queda extraño la combinación de ambos.

  No incluye footer.

  \subsection{Prototipo final}

  Para el prototipo final se ha decidido crear uno nuevo uniendo los puntos positivos de cada uno de los prototipos de la fase divergente.

  \vspace{2mm}

  Del prototipo 1 se ha decidido implementar la \textbf{gama de colores} azul-blanco, aunque se ha quitado los gradientes de colores para lograr un estilo más profesional. A su vez también se ha implementado el \textbf{menú izquierdo} que dispone el usuario en su perfil.

  \vspace{2mm}

  Del prototipo 2 se ha implementado la \textbf{estructura del diseño de la página principal}, así como el diseño del \textbf{header} y la estructura de diseño del \textbf{footer}.

  \vspace{2mm}

  Del prototipo 3 se ha decidido incluir la \textbf{estructura de tarjetas} para los servicio en la aplicación, así como la tarjeta que representa el \textbf{perfil del usuario}.

  \vspace{2mm}

  Por último, del prototipo 4 se ha implementado la \textbf{barra de navegación} debajo del header, así como las \textbf{etiquetas de búsqueda} y el diseño final de las \textbf{tarjetas de servicios}.

  \vspace{2mm}

  Para este prototipo final, se han tenido en cuenta los requisitos redactados en la entrega anterior.

  \vspace{2mm}

  Dentro de los \textbf{requisitos funcionales} se han implementado los relacionados a la gestión de la cuenta (creación, inicio sesión, cerrar sesión), así como los relacionados con el perfil de la cuenta y sus ajustes (acceso a perfil, actualización de perfil).

  \vspace{2mm}

  También se han incluido los relacionados a los servicios (solicitud y publicación) además de los relacionados con la búsqueda, la comunicación entre usuarios y la valoración de los mismos.

  \vspace{2mm}

  También se ha tenido en cuenta el requisitos de convertirse en \textbf{h3lper} a través de publicar servicios y la implementación de la moneda \textbf{H3lp Coin} para la tienda.

  Por otra parte, los \textbf{requisitos no funcionales} se han implementado en el diseño de la aplicación directamente, entre los que se destacan los relacionados con el entorno (plataforma, navegador y multidispositivo), la creación de cuentas, a frecuencia de uso y la experiencia de usuario.

  \vspace{2mm}

  Por último el requisito de Fuente ha sido modificado puesto que se ha elegido como nueva fuente para la aplicación: \textit{Helvetica Neue}.

  \section{Diseño Final: Escenario y Detalles de Diseño}

  En las siguientes imágenes se muestra el storyboard donde una persona necesita que alguien cuide a sus hijos. Este es un escenario de uso de la aplicación y abarca desde que surge la necesidad de la persona, hasta que queda solucionada a través de la aplicación web.


  \imgcenter[110]{Images/storyboard1.png}

  En la imagen superior, se observa la pantalla de inicio diseñada. Cuenta con un \textit{header} donde se muestra el logo de \textit{H3lp Me}, el nombre y dos botones tanto para registrarse como para iniciar sesión.  Se incorpora un buscador de servicios y varias categorías a las que acceder de forma directa para que cualquier usuario que entre en la web (aunque no esté registrado) pueda consultar lo que \textit{H3lp Me} ofrece, por lo que puede ayudar a captar usuarios si ven la aplicación útil.

  \vspace{2mm}

  Al tratarse de la página de inicio, aparecen varias frases sobre la propuesta de valor de \textit{H3lp Me} y un vídeo de explicación de \textit{H3lp Me}. De forma que se permite a los nuevos usuarios tener una concepción de la aplicación desde el primer momento en que entran en la web. Además, se incorpora un footer donde se muestra información general de la H3lp Me. Para que, desde un principio el usuario conozca quién está detrás de la página y cómo contactar.

  \imgcenter[110]{Images/storyboard2.png}

  Una vez que el usuario se ha registrado y ha accedido a su cuenta, aparecerá en la página de la imagen superior, en la que se puede ver un conjunto de tarjetas que representan los servicios que otros usuarios ofrecen o solicitan. Para que sea más rápido, se incorpora un botón en la parte derecha superior para que el usuario pueda ofrecer un servicio. Tanto el \textit{footer} como el \textit{header} se mantienen a lo largo de la página. Sin embargo, en el \textit{header} se incorpora la la foto de perfil del usuario, el nombre y el nivel del usuario (explicado en la entrega de gamificación) y una rueda que simboliza los ajustes a los que el usuario puede acceder. Por lo que esta es la solución propuesta para que cualquier usuario vea servicios y que el propio usuario pueda añadir y solicitar los servicios.

  \imgcenter[110]{Images/storyboard3.png}

  Para acceder a la pantalla superior, como se ha indicado en el \textit{storyboard}, el usuario debe pulsar en su foto o nombre de perfil. En dicha página hay un menú en la parte izquierda con diversas funcionalidades, la primera de ellas es "Tus Categorías", donde se muestran las categorías en las que estás interesado. La segunda es "Tus servicios", donde puedes consultar aquellos servicios solicitados y ofrecidos, así como añadir un nuevo servicio. La tercera es la sección de tienda, que se explica más adelante. Y, por último, una funcionalidad de chat para que así los usuarios que contraten un servicio puedan comunicarse con la persona que lo ofrece y acordar más detalles para llevarlo a cabo como hora, precio (si no se ha establecido ninguno) y cualquier cosa que necesiten hablar.

  \vspace{2mm}

  En esta misma pantalla, en la parte derecha hay una tarjeta que muestra de forma resumida la información del usuario: nombre de usuario, monedas que tiene, nivel, foto de perfil, valoración como \textit{H3lper} y valoración como usuario (estas características se explican en el trabajo de gamificación).

  \vspace{2mm}

  En la pantalla superior, se ve la sección de tienda que aparece en el menú. En ella el usuario podrá comprar con las \textit{H3lp Coins} (monedas de la web) promociones para 1 o varios servicios y así aparecer en los más recomendados para ganar visibilidad.

  \vspace{2mm}

  Aquí aparece la sección de chats, en la que el usuario podrá iniciar un chat con las personas que contraten sus servicios. Se muestra en forma de \textit{pop-up} y es la base para comunicarse entre usuarios, el que ofrece servicios y el que los recibe.

  \vspace{2mm}

  Cuando el usuario está en la página de inicio, si pincha en el botón “+”, le aparece la página superior, en la que el usuario puede ver información detallada sobre el servicio al que ha clickado. Desde aquí, podrá contratarlo e iniciar un chat con el usuario que ofrece el servicio. Si lo desea puede volver hacia la página donde aparecen todos los servicios, pinchando en la flecha situada en la parte superior izquierda.


  A continuación, se muestran los wireframes de la aplicación web.


  \imgcenter[110]{Images/inicio1.png}

  \imgcenter[110]{Images/iniciosesionhelpme.png}

  \imgcenter[110]{Images/cuid.png}

  \imgcenter[110]{Images/serviciosinmenu.png}

  \imgcenter[110]{Images/contratopop.png}

  \imgcenter[110]{Images/contratado.png}

  \imgcenter[110]{Images/ajustes.png}

  \imgcenter[110]{Images/tienda.png}

  \imgcenter[110]{Images/perfildsi.png}

  \imgcenter[110]{Images/chats.png}




\end{document}
