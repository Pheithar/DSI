\documentclass{uc3mpracticas}



%%%%%%%%%%%%%%%%%%%%%%%%%%%%%%%%%%%%%%%%%%%%%%%%%%%%%%%%%%%%%%%%%%%%%%%%%%%%%%%%
%%%                   Plantilla Prácticas UC3M                               %%%
%%%                Universidad Carlos III de Madrid                          %%%
%%%                   Alejandro Valverde Mahou                               %%%
%%%%%%%%%%%%%%%%%%%%%%%%%%%%%%%%%%%%%%%%%%%%%%%%%%%%%%%%%%%%%%%%%%%%%%%%%%%%%%%%

%Permitir cabeceras y pie de páginas personalizados
\pagestyle{fancy}

%Path por defecto de las imágenes
\graphicspath{ {./images/} }

%Declarar formato de encabezado y pie de página de las páginas del documento
\fancypagestyle{doc}{
  %Cabecera
  \headerpr[1]{Fase de Implementación}{\textbf{}}{Diseño de Sistemas Interactivos}
  %Pie de Página
  \footerpr{\textbf{Universidad Carlos III de Madrid}}{}{{\thepage} de \pageref{LastPage}}
}

%Declarar formato de encabezado y pie del título e indice
\fancypagestyle{titu}{%
  %Cabecera
  \headerpr{}{}{}
  %Pie de Página
  \footerpr{}{}{}
}


\appto\frontmatter{\pagestyle{titu}}
\appto\mainmatter{\pagestyle{doc}}


\begin{document}
  %Comienzo formato título
  \frontmatter


  \centeredtitle{Images/logo_help.png}{Grado en Ingeniería Informática}{Curso 2020}{Diseño de Sistemas Interactivos}{Fase de Implementación: H3lp Me}

    \vspace{2mm}

    \begin{center}
      \line(1, 0){450}
    \end{center}

    \authorsright{Alejandro Parrado Arribas}{100383453}{Adrián Sanz Gómez}{100383473}{Alejandro Valverde Mahou}{100383383}{Andrés Vinagre Blanco}{100383414}

    \newpage


    %Índice
    \tableofcontents

  \newpage

  %Comienzo formato documento general
  \mainmatter


  \section{Introducción}

  Este documento sirve como ayuda complementaria para la entrega del código de la página de \textit{h3lp me}. Este documento se divide en varios apartados:
  \begin{itemize}
    \item \textbf{Tecnología Empleada}: Este apartado aportará una explicación del diseño y la forma que ha sido implementado.
    \item \textbf{Productos Generados}: Este apartado aportará una breve descripción del producto que se ha creado al finalizar esta práctica.
    \item \textbf{Manual de Despliegue}: Este apartado informará de la locaclización de la página web y los recursos que usa, como puede ser información de la base de datos o el servidor.
    \item \textbf{Manual de Uso}: Este apartado explicará el funcionamiento de la página desde el punto de vista del usuario.
    \item \textbf{Distribución de Trabajo}: En este apartado se indiacrá el número de horas y la carga de trabajo realizada por cada uno de los miembros del equipo.
  \end{itemize}

  \section{Tecnología Empleada}

  La tecnología en la que se ha desarrollado la Práctica es \textbf{Angular 9}. Se ha optado por utilizar una programación basada en componentes donde cada componente cumple una función específica. Los componentes usados son:

  \begin{itemize}
    \item \textbf{add}

      Este componente es el encargado de mostrar cada uno de los anuncios publicados en la página. Para acceder a cada uno de ellos, se tiene que acceder desde la ventana de \textit{servicios}, y pulsar en el botón de ver más de cada anuncio.

    \item \textbf{card}

      Este componente es cada uno de los anuncios que se muestran en la ventana de \textit{servicios}. Es una versión reducida de la ventana de \textit{add}.

    \item \textbf{footer}

      Es el pie de página de la aplicación. Es visible desde todas las ventanas y tiene un caracter principalmente informativo. En él están indicados las referencias de las imágenes de la página y enlace a los creadores.

    \item \textbf{header}

      Es la cabecera de la página, y es visible desde cualquier ventana de ella. En ella está el componente \textit{searcher}, donde realzian las búsquedas de servicios.

      \vspace{2mm}

      Además en ella están los botones de registro e inicio de sesión, con su pop-up correspondiente. Si el usuario ya ha iniciado sesión, esos botones se sustituyen por el símbolo de su perfil.

    \item \textbf{main}

      Es el componente que se encuentra ne la página principal, y posee sobre todo elementos visuales.

    \item \textbf{not-found}

      Es un componente auxiliar que se muestra si el usuario intenta escribir a mano la url y se equivoca. Permite redirigirse a la página principal.

    \item \textbf{profile}

      Es la ventan del perfil de cada usuario. Si esa ventana corresponde con la del usuario registrado, permite hacer acciones adicionales, cmo crear un servicio nuevo si el usuario es \textit{h3lper}.

    \item \textbf{searcher}

      Este componente permite realizar las búsquedas por nombre o por categoría (dentro de las categorias predeterminadas).

    \item \textbf{servicios}

      Es la ventana mostrada al realizar una búsqueda, y muestra todos los servicios que coinciden con esa búsqueda.

  \end{itemize}

  Además de los componentes, se usan dos servicios para controlar y manjear la aplicación:

  \begin{itemize}
    \item \textbf{FireStore}

      Este servicio se encarga de alojar las funciones necesarias para establecer y realizar la comunicación con la base de datos de \textit{FireBase}.

    \item  \textbf{Global}

      Es un servicio que contiene el usuario que ha iniciado sesión actualmente. Esto sirve para que la sesión no expire cuando el usuario navega entre las diferentes ventanas de la aplicación.

  \end{itemize}

  Por último, se han creado dos objetos auxiliares para el tratamiento de datos:

  \begin{itemize}
    \item \textbf{User}

        Tiene los campos de los usuarios.

    \item \textbf{Advertisement}

      Tiene los campos de los anuncios.

  \end{itemize}

  \section{Productos Generados}

  \section{Manual de Despliegue}

  \section{Manual de Uso}

  \section{Distribución del Trabajo}






\end{document}
