\documentclass{uc3mpracticas}



%%%%%%%%%%%%%%%%%%%%%%%%%%%%%%%%%%%%%%%%%%%%%%%%%%%%%%%%%%%%%%%%%%%%%%%%%%%%%%%%
%%%                   Plantilla Prácticas UC3M                               %%%
%%%                Universidad Carlos III de Madrid                          %%%
%%%                   Alejandro Valverde Mahou                               %%%
%%%%%%%%%%%%%%%%%%%%%%%%%%%%%%%%%%%%%%%%%%%%%%%%%%%%%%%%%%%%%%%%%%%%%%%%%%%%%%%%

%Permitir cabeceras y pie de páginas personalizados
\pagestyle{fancy}

%Path por defecto de las imágenes
\graphicspath{ {./images/} }

%Declarar formato de encabezado y pie de página de las páginas del documento
\fancypagestyle{doc}{
  %Cabecera
  \headerpr[1]{Fase de Implementación}{\textbf{}}{Diseño de Sistemas Interactivos}
  %Pie de Página
  \footerpr{\textbf{Universidad Carlos III de Madrid}}{}{{\thepage} de \pageref{LastPage}}
}

%Declarar formato de encabezado y pie del título e indice
\fancypagestyle{titu}{%
  %Cabecera
  \headerpr{}{}{}
  %Pie de Página
  \footerpr{}{}{}
}


\appto\frontmatter{\pagestyle{titu}}
\appto\mainmatter{\pagestyle{doc}}


\begin{document}
  %Comienzo formato título
  \frontmatter


  \centeredtitle{Images/logo_help.png}{Grado en Ingeniería Informática}{Curso 2020}{Diseño de Sistemas Interactivos}{Fase de Implementación: H3lp Me}

    \vspace{2mm}

    \begin{center}
      \line(1, 0){450}
    \end{center}

    \authorsright{Alejandro Parrado Arribas}{100383453}{Adrián Sanz Gómez}{100383473}{Alejandro Valverde Mahou}{100383383}{Andrés Vinagre Blanco}{100383414}

    \newpage


    %Índice
    \tableofcontents

  \newpage

  %Comienzo formato documento general
  \mainmatter


  \section{Introducción}

  Este documento sirve como ayuda complementaria para la entrega del código de la página de \textit{h3lp me}. Este documento se divide en varios apartados:
  \begin{itemize}
    \item \textbf{Tecnología Empleada}: Este apartado aportará una explicación del diseño y la forma que ha sido implementado.
    \item \textbf{Productos Generados}: Este apartado aportará una breve descripción del producto que se ha creado al finalizar esta práctica.
    \item \textbf{Manual de Despliegue}: Este apartado informará de la locaclización de la página web y los recursos que usa, como puede ser información de la base de datos o el servidor.
    \item \textbf{Manual de Uso}: Este apartado explicará el funcionamiento de la página desde el punto de vista del usuario.
    \item \textbf{Distribución de Trabajo}: En este apartado se indiacrá el número de horas y la carga de trabajo realizada por cada uno de los miembros del equipo.
  \end{itemize}

  \section{Tecnología Empleada}

  La tecnología en la que se ha desarrollado la Práctica es \textbf{Angular 9}. Se ha optado por utilizar una programación basada en componentes donde cada componente cumple una función específica. Los componentes usados son:

  \begin{itemize}
    \item \textbf{add}

      Este componente es el encargado de mostrar cada uno de los anuncios publicados en la página. Para acceder a cada uno de ellos, se tiene que acceder desde la ventana de \textit{servicios}, y pulsar en el botón de ver más de cada anuncio.

    \item \textbf{card}

      Este componente es cada uno de los anuncios que se muestran en la ventana de \textit{servicios}. Es una versión reducida de la ventana de \textit{add}.

    \item \textbf{footer}

      Es el pie de página de la aplicación. Es visible desde todas las ventanas y tiene un caracter principalmente informativo. En él están indicados las referencias de las imágenes de la página y enlace a los creadores.

    \item \textbf{header}

      Es la cabecera de la página, y es visible desde cualquier ventana de ella. En ella está el componente \textit{searcher}, donde realzian las búsquedas de servicios.

      \vspace{2mm}

      Además en ella están los botones de registro e inicio de sesión, con su pop-up correspondiente. Si el usuario ya ha iniciado sesión, esos botones se sustituyen por el símbolo de su perfil.

    \item \textbf{main}

      Es el componente que se encuentra ne la página principal, y posee sobre todo elementos visuales.

    \item \textbf{not-found}

      Es un componente auxiliar que se muestra si el usuario intenta escribir a mano la url y se equivoca. Permite redirigirse a la página principal.

    \item \textbf{profile}

      Es la ventan del perfil de cada usuario. Si esa ventana corresponde con la del usuario registrado, permite hacer acciones adicionales, cmo crear un servicio nuevo si el usuario es \textit{h3lper}.

    \item \textbf{searcher}

      Este componente permite realizar las búsquedas por nombre o por categoría (dentro de las categorias predeterminadas).

    \item \textbf{servicios}

      Es la ventana mostrada al realizar una búsqueda, y muestra todos los servicios que coinciden con esa búsqueda.

  \end{itemize}

  Además de los componentes, se usan dos servicios para controlar y manjear la aplicación:

  \begin{itemize}
    \item \textbf{FireStore}

      Este servicio se encarga de alojar las funciones necesarias para establecer y realizar la comunicación con la base de datos de \textit{FireBase}.

    \item  \textbf{Global}

      Es un servicio que contiene el usuario que ha iniciado sesión actualmente. Esto sirve para que la sesión no expire cuando el usuario navega entre las diferentes ventanas de la aplicación.

  \end{itemize}

  Por último, se han creado dos objetos auxiliares para el tratamiento de datos:

  \begin{itemize}
    \item \textbf{User}

        Tiene los campos de los usuarios.

    \item \textbf{Advertisement}

      Tiene los campos de los anuncios.

  \end{itemize}

  \section{Manual de Despliegue}

  Los servicios y programas utilizados para la creación de la página web han sido los siguientes:

  \begin{itemize}
    \item \textbf{Angular 9}: Se ha usado para desarrollar toda la aplicación. Es la aplicación con la que se pedía hacer la práctica. Para añadir funcionalidad adicional, se usan cookies, que sirven para mantener la sesión del usuario que ha iniciado sesión, de forma que la sesión de un usuario dura hasta que cierra sesión manualmente, o se cierra el navegador completamente.
    \item \textbf{Firebase Database}: Se han usado dos colecciones creadas en esta base de datos: \textit{Usuarios} y \textit{Servicios}, para almacenar la información necesaria de la aplicación.
    \item \textbf{Firebase Storage}: Se usa el amacenamiento de firebase para almacenar imágenes, tanto de los usuarios como de los servicios.
    \item \textbf{Firebase Hosting}: Se usa para desplegar la aplicación en la red, de forma que sea accesible desde los navegadores web.
    \item \textbf{Electron}: Se usa para generar aplicaciones de escritorio para los sistemas operativos Linux, Windos y MacOS.
  \end{itemize}

  \section{Manual de Uso}

  Nuestra aplicación H3lp Me está enfocada a navegadores web como Firefox y Google Chrome. Es importante que para probar todas las funcionalidades se siga esta guía.

  \vspace{2mm}
  \imgcenter{Images/'HELPME (1).jpg'}
  Nada más entrar a la aplicación, nos encontramos con la página de inicio. En esta primera página corresponde a la vista que tiene un usuario al entrar a la aplicación.

  En la parte superior, tenemos un header con el logo de la aplicación que si lo pulsamos te lleva a esta página, a la derecha están los botones registrarse e iniciar sesión. Justo debajo hay una barra de navegación para buscar servicios que ofrezca la gente y unos filtros ya preestablecidos como puede ser fotografía, informática, cocina etc. Esto permite que gente que no esté Logueada pueda buscar servicios que le interesen.

  En el cuerpo de la página hay un botón de registrarse que sirven para los nuevos usuarios y en la parte inferior se encuentra el footer con los perfiles de github de los desarrolladores.

  \vspace{2mm}
  \imgcenter{Images/'HELPME (2).jpg'}
  Cuando un usuario se va a registrar, tiene que pulsar el botón de registrarse y aparecerá un popup en la pantalla. Se solicitarán los campos de nombre de usuario, contraseña, repetir contraseña y quiero ser el h3lper. Es importante en este paso seleccionar la opción de Quiero ser h3lper y recordar el nombre y contraseña. Para poder rellenar estos campos se necesita llegar al mínimo de caracteres. Los usuarios que decidan ser h3lper podrán publicitar sus servicios en la aplicación.

  \vspace{2mm}
  \imgcenter{Images/'HELPME (3).jpg'}
  Inmediatamente cuando una persona se registre, irá directamente a su perfil en la aplicación en el podrá ver sus servicios solicitados, que inicialmente serán cero y una tienda donde puede seleccionar promociones premium para posicionar los anuncios.

  \vspace{2mm}
  \imgcenter{Images/'HELPME (1).png'}
  \imgcenter{Images/'HELPME (4).jpg'}
  \imgcenter{Images/'HELPME (5).jpg'}
  A la derecha aparece el perfil del usuario, lo primero que hay es su foto, puede ser cambiada por una que tenga en su ordenador, para ello necesitas pinchar encima de ella y seleccionar la que quieras. También aparece su nombre, monedas para solicitar promociones en la tienda, un galardón que te identifica como h3lper y una barra de experiencia con niveles que tiene el usuario.

  En el header, como tenemos las sesión iniciada aparece nuestra foto de perfil, nombre y nivel. También, una rueda de configuración que nos permite cambiar el nombre de usuario, cambiar la contraseña y cerrar sesión.

  \vspace{2mm}
  \imgcenter{Images/'HELPME (6).jpg'}
  \imgcenter{Images/'HELPME (7).jpg'}
  Como nos registramos como h3lper, en nuestro perfil también aparecerá una sección llamada tus servicios ofrecidos. En ella saldrán los anuncios que el usuario publicita en nuestra aplicación. También encontramos un botón que nos permite añadir un nuevo servicio. Al pulsarlo se nos abrirá un popup donde nos solicitan los campos necesarios para publicar un anuncio tales como nombre del servicio, categoría, una descripción opcional, localización, precio y una imagen que tenga el usuario. Es importante que se creen DOS anuncios en este punto.

  \vspace{2mm}
  \imgcenter{Images/'HELPME (8).jpg'}
  Tras crear los anuncios, nos darán 30 puntos de experiencia por cada uno y nos aparece en la esquina superior derecha un número que simboliza las personas que han solicitado el servicio publicado.

  \vspace{2mm}
  \imgcenter{Images/'HELPME (9).jpg'}
  \imgcenter{Images/'HELPME (10).jpg'}
  A continuación nos vamos a ir a solicitar servicios, para ello vamos a dar uso de la barra que hay debajo del header. Para continuar con la guía es necesario solicitar 4 servicios, es recomendable pulsar en ver todos los servicios. Cuando veas todos los anuncios se puede ver como aparecen en la parte superior los servicios promocionados para que los vean más gente, debajo están el resto. A continuación se van a solicitar servicios, para esto se debe pulsar el botón que ahí abajo a la derecha en los anuncios +INFO. Se nos abrirá la información detallada del anuncio como puede ser categoría, precio, localización etc. Abajo está el botón que solicitar servicio. Al pulsarlo solicitaremos el servicio y ganaremos 10 puntos de experiencia. Ese necesario repetir este paso 8 veces más para que consigamos el nivel dos y nos ganemos las primeras monedas. Aquí se puede aprovechar para echar un vistazo a la barra de búsqueda y en los distintos filtros que aparecen en la parte superior del aplicación como fotografía, informática etc.

  Tras realizar esta tarea, nos vamos a dirigir a nuestro perfil pulsando en nuestra imagen perfil que hay en el header. Como veremos hemos conseguido el nivel dos y algunas monedas.

  \vspace{2mm}
  \imgcenter{Images/'HELPME (11).jpg'}
  \imgcenter{Images/'HELPME (12).jpg'}
  Ahora nos dirigiremos al apartado de tienda, aquí solicitaremos la promoción de un día que cuesta 10 monedas. Se nos abrirá un popup con los anuncios que tenemos publicados, donde aparecerá el nombre del anuncio, precio y fecha de creación. A la derecha de eso hay un activable que pulsándolo habremos seleccionado el anuncio a promocionar, para confirmar hay que pulsar el botón de obtener promoción.

  \vspace{2mm}
  \imgcenter{Images/'HELPME (13).jpg'}
  Para comprobar esta acción basta con pulsar en el botón de ver todos los servicios y veremos que el anuncio aparece en la sección de servicios promocionados.

  \vspace{2mm}
  \imgcenter{Images/'HELPME (14).jpg'}
  La otra vista de la aplicación corresponde a las personas que no son h3lper, para ello vamos a cerrar la sesión pulsando en la rueda de configuración que se encuentra en el header.

  \vspace{2mm}
  \imgcenter{Images/'HELPME (15).jpg'}
  Apareceremos en la página de inicio y nos registraremos de nuevo, pero en este caso marcando en el último campo que no quiero ser h3lper.

  \vspace{2mm}
  \imgcenter{Images/'HELPME (16).jpg'}
  \imgcenter{Images/'HELPME (17).jpg'}
  Una vez registrados, en el perfil notaremos que no tenemos un apartado de servicios ofrecidos. Vamos a dirigirnos al anuncio que hemos promocionado con la otra cuenta y lo solicitaremos.

  \vspace{2mm}
  \imgcenter{Images/'HELPME (18).jpg'}
  Volvemos a cerrar sesión y nos Logueamos con la primera cuenta que hemos creado.

  \vspace{2mm}
  \imgcenter{Images/'HELPME (19).jpg'}
  \imgcenter{Images/'HELPME (20).jpg'}
  En nuestro perfil, en el apartado de servicios ofrecidos veremos en la esquina superior derecha que aparece 1. Esto significa que una persona ha solicitado el servicio. Entrando al anuncio pulsando en el boton +INFO, se pueden ver los nombres de las personas que han solicitado nuestro servicio.





  \section{Distribución del Trabajo}






\end{document}
