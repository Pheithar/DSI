\documentclass{uc3mpracticas}



%%%%%%%%%%%%%%%%%%%%%%%%%%%%%%%%%%%%%%%%%%%%%%%%%%%%%%%%%%%%%%%%%%%%%%%%%%%%%%%%
%%%                   Plantilla Prácticas UC3M                               %%%
%%%                Universidad Carlos III de Madrid                          %%%
%%%                   Alejandro Valverde Mahou                               %%%
%%%%%%%%%%%%%%%%%%%%%%%%%%%%%%%%%%%%%%%%%%%%%%%%%%%%%%%%%%%%%%%%%%%%%%%%%%%%%%%%

%Permitir cabeceras y pie de páginas personalizados
\pagestyle{fancy}

%Path por defecto de las imágenes
\graphicspath{ {./images/} }

%Declarar formato de encabezado y pie de página de las páginas del documento
\fancypagestyle{doc}{
  %Cabecera
  \headerpr[1]{Análisis y Estudio de Campo}{\textbf{H3lpMe}}{Diseño de Sistemas Interactivos}
  %Pie de Página
  \footerpr{}{}{{\thepage} de \pageref{LastPage}}
}

%Declarar formato de encabezado y pie del título e indice
\fancypagestyle{titu}{%
  %Cabecera
  \headerpr{}{}{}
  %Pie de Página
  \footerpr{}{}{}
}


\appto\frontmatter{\pagestyle{titu}}
\appto\mainmatter{\pagestyle{doc}}


\begin{document}
  %Comienzo formato título
  \frontmatter


  %Portada 1 (Centrado todo)
  \centeredtitle{Images/LogoUC3M.png}{Grado en Ingeniería Informática}{Curso 2020}{Diseño de Sistemas Interactivos}{Análisis y Estudio de Campo: H3lpMe}

  \vspace{40mm}

  \begin{center}
    \line(1, 0){450}
  \end{center}

  \authorsright{Alejandro Parrado Rivas}{100383453}{Adrián Sanz Gómez}{100383473}{Alejandro Valverde Mahou}{100383383}{Andrés Vinagre Blanco}{100383414}

  \newpage


  %Índice
  \tableofcontents

  \newpage

  %Comienzo formato documento general
  \mainmatter

  \section{Introducción}

  \textbf{H3lpMe} es una aplicación web a través la cual los usuarios pueden pedir ayuda a otros en cualquier ámbito de la vida cotidiana o profesional, como puede ser un servicio de comida, resolver un problema informático, o cualquier incidente que pueda surgir en el día a día.

  \vspace{4mm}

  El objetivo fundamental de la aplicación es hacer más fácil el día a día de las personas, ayudando en aquellas actividades en las que los usuarios no pueden desenvolverse, sin tener la necesidad de contratar a un profesional o realizar trámites complejos a través de intermediarios.

  \vspace{4mm}

  En la app, cualquier persona puede crearse un perfil indicando sus habilidades y capacidades, para que otros usuarios de la plataforma puedan solicitarlas cuando lo necesiten acordando un precio entre ellos.

  \section{Contexto de Diseño}

  En un estudio previo, se pueden definir dos grupos de \textit{usuarios} principales:

  \vspace{4mm}

  El primer grupo está formado por personas con una edad comprendida entre los 25 y los 50 años. Este grupo de usuarios ofrecerá y solicitará servicios que tengan que ver con el cuidado y mantenimiento de un hogar, como puede ser servicio de reparaciones, limpieza o cocina. Su conocimiento de los sistemas informáticos y aplicaciones no es experto, pero los conocen y se pueden manejar con facilidad con ellos.

  \vspace{4mm}

  El segundo grupo está formado po personas con una edad comprendida entre los 16 y los 25 años. El propósito que le dan a estos usuarios a la aplicación estará más centrado a los estudios, por lo que los servicios que ofrecerán o solicitarán será principalmente el de las clases particulares, pudiendo también ofrecer servicios de cuidado de niños pequeños u potras tareas, pero normalmente no solicitándolas.

  % ESTO NECESITA UNA REVISIÓN Y SEGUNDA REDACCIÓN


  \vspace{8mm}

  Como \textit{stakeholders} se pueden definir varios grupos:

  \begin{itemize}
    \item \textbf{Usuarios finales de la aplicación}, que ya hemos dividido previamente en dos grupos.
    \item \textbf{Programadores y desarrolladores}, que son las personas ivolucradas en el desarrollo de ka aplicación. Deben conocer el funcionamiento de la aplicación completa, y son los encargados de que se cumplan las espectativas de los usuarios.
    \item \textbf{Diseñadores}, que necesitan realizar prototipos y diseños que cuplan las necesidades de los usuarios, y sean intuitivos y completos.
  \end{itemize}


  \vspace{8mm}



  \section{Diseño y Desarrollo de Técnicas de Investigación del Trabajo de Campo: Obtención de Datos}

  Debido al estado de alarma en el que se encuentra el pais donde estamos relizando este estudio, hemos decidido usar dos técnicas de investigación que nos permiten la obtención de datos de manera telemática. Por esta razón hemos optado por un \textbf{cuestionario online} y \textbf{documentación y estado del arte} (\textit{state of art}).

  Por una parte, el cuestionario online ha sido desarrollado a través de la herramienta: \textit{Formularios de Google}. Esto permite una gran difusión en poco tiempo.

  Además, se realizará una investigación sobre el estado actual del arte, las aplicaciones actuales que ofrecen un servicio similar, y las posibles leyes que habrá que tener en cuenta a la hora de ofrecer los servicios de la aplicación al público.

  \subsection{Cuestionario}

  El cuestionario posee las siguientes preguntas para los potenciales usuarios:

  \begin{itemize}
    \item \textbf{¿Cuántos años tienes?}:
      Esta pregunta permite conocer el rango de edad de los posibles usuarios

    \item \textbf{¿Has usado alguna vez Wallapop o aplicaciones similares de compra/venta?}:
      Esta pregunta permite conocer la familiaridad que tienen los posibles usuarios con aplicaciones que ofrecen un servicio parecido, ya que, al igual que la aplicación a desarrollar, se basan en la confianza entre desconocidos para, en el caso de \textit{Wallapop}, la compra-venta de productos, y en el caso de la aplicación a desarrollar, el ofrecimiento de servicios.

    \item \textbf{¿Has usado alguna vez aplicaciones como Just Eat, Uber Eats o similares de comida a domicilio?}:
      Esta pregunta permite conocer la familiaridad que tienen los posibles usuarios con aplicaciones que ofrecen servicios a domicilio, ya que la aplicación a desarrollar también posee este servicio.

    \item \textbf{¿Alguna vez has necesitado un profesor particular/fontanero/técnico informático/pintor o albañil/niñero o cuidador?}:
      Esta pregunta permite conocer las necesidades de los posibles usuarios (cada '/' supone una pregunta diferente, con el mismo formato).

    \item \textbf{En caso afirmativo, ¿cómo le conociste?}:
      Esta pregunta permite conocer como los poibles usuarios suelen buscar actualmente gente que ofrezca los servicios que necesitan.

    \item \textbf{¿Cuáles de los siguientes servicios te ves capacitado a ofrecer?}:
      Esta pregunta facilita la identificación de posibles usuarios que ofrecerán servicios dentro de la aplicación, y cuales son los servicios que podrán ofrecer (las opciones facilitadas son : \textit{Profesor Particular}, \textit{Fontanero}, \textit{Técnico Informático}, \textit{Pintor}, \textit{Niñero}, \textit{Cuidador}, \textit{Albañil}, \textit{Cocinero}, \textit{Limpieza del hogar}, \textit{Mudanzas}, \textit{Fotógrafo}).

    \item \textbf{¿Confiarías en alguien que no conoces para realizar alguno de los servicios mencionados anteriormente?}:
      Esta pregunta permite conocer si los posibles usuarios que requieran de servicios aceptarían a otros usuarios que ofrecieran esos servicios en sus hogares.

    \item \textbf{¿Si buscaras trabajo en alguna de las áreas anteriores o similares, te registrarías en una aplicación donde soliciten trabajos puntuales a domicilio?}:
      Esta pregunta permite identificar posibles usuarios que ofrecerían servicios a través de la aplicación.

    \item \textbf{¿Si necesitaras solicitar un trabajo puntual en alguna de las áreas anteriores o similares, te registrarías en una aplicación donde personas de diferente experiencia ofrezcan sus servicios?}:
      Esta pregunta permite identificar a los posibles usuarios que solicitarían servicios a través de la aplicación.

  \end{itemize}

  Este cuestionario ha sido desarrollado y creado por los 4 miembros del equipo, sin la asignación de ningún rol específico a parte del de la difusión del cuestionario por diversas plataformas y contactos.

  Con las preguntas buscamos conocer la opinión de los diferentes usuarios potenciales a los que pueda interesar la aplicación, así como poder identificar los perfiles de los usuarios primarios, secundarios y terciarios que pudiera tener. Dentro de los datos, esperamos obtener las respuestas de personas dentro de un rango de edad potencial para nuestra aplicación \textbf{(16-70)} para poder realizar un análisis completo y así diseñar la aplicación depurándola con las necesidades de los usuarios.

  Dentro de los datos que esperamos obtener, entendemos que no todos serán a favor de la aplicación, siendo del todo respetable y dentro de lo razonable. A pesar de ello, prescindiremos de los datos de los cuales no podamos obtener ideas o mejorar el software \textbf{(usuarios no potenciales)}, así como de las respuestas aparentemente falsas.

  \subsection{Documentacion y estado del arte}

  Para establecer un contexto y estado del arte apropiados hemos decidido elegir 3 aplicaciones y sitios web que implementan parcialmente las funciones de nuestro software. Estas son \textbf{Wallapop}, que ofrece un servicio de compra venta entre usuarios sin intermediarios; \textbf{JustEat}, un servicio que actua de intermediario entre restaurantes y usuarios para ofrecer comida a domicilio; y \textbf{Milanuncios}, un sitio web que ofrece una amplia de servicios y productos en diferentes areas, siendo esta web la que más se asimila a nuestro software.

    \subsubsection{Wallapop}

      Wallapop es una empresa española fundada en 2013, que ofrece una plataforma dedicada a la compra venta de productos de segunda mano entre usuarios a través de internet. La plataforma es accesible desde dispositivos móviles \textit{(Android e iOS)} y desde el sitio web online.

      Algunas de las características de Wallapop son:

      \begin{itemize}
        \item Chat integrado
          Gracias a este chat los compradores pueden ponerse en contacto con los vendedores para preguntar cosas, regatear, quedar...
        \item Busqueda por cercanía (geolocalización)
          Gracias a la geolocalización del movil el usuario puede buscar anuncios que han publicado cerca de su zona
        \item Enfocado al mercado de segunda mano
          La mayoria de los anuncios publicados son objetos de sengunda mano pero tambien se pueden encontrar cosas nuevas o gente ofreciendo servicios
        \item Los vendedores son particulares o profesionales
          Los vendedores pueden ser particulares o profesionales, estos ultimos son faciles de reconocer dentro de la aplicacion porque cuentan con un tick de verificado
        \item Envios a través de la aplicación
          Wallapop tiene un convenio con Correos y cuenta con un servicio de envios a traves de la aplicación para evitar estafas
        \item Valoraciones
          Cuando se vende un producto, el vendedor puede recibir una valoracion por parte del comprador con 5 estrellas


      \end{itemize}

    \subsubsection{JustEat}

    JustEat es una compañia de servicios fundada en 2001 dedicada a la distribución de comida a domicilio en varios formatos. Actua como intermediario entre los clientes y bares o restaurantes, cobrando una comisión por cada pedido. La plataforma es accesible desde su sitio web y a través de dispositivos móviles \textit{(Android e iOS)}.

      \begin{itemize}
        \item Busqueda por cercania
          Para realizar una busqueda lo primero que debes introducir es una direccion y te muestre los restaurantes mas cercanos que hay en tu zona
        \item Enfocado a comnida a domicilio
          Lo que ofrece esta aplicacion
        \item Los vendedores son profesionales
        \item Envios a través de la aplicación
        \item Valoraciones

      \end{itemize}

    \subsubsection{Milanuncios}

    Milanuncios es una empresa española creada en 2005. Es una plataforma en la cual un usuario puede poner un anuncio de cualquier area ofreciendo servicios o productos, siempre dentro de unos ámbitos legales y aceptados por la plataforma. Es accesible a través de su sito web y dispositivos móviles \textit{(Android e iOS)}.

      \begin{itemize}
        \item Chat integrado
        \item Busqueda por cercania
        \item Enfocado al mercado de segunda mano, empleo
        \item Los vendedores son particulares o profesionales
        \item Valoraciones

      \end{itemize}



  % REVISAR
  Como ya hemos indicado antes, por cuestiones excepcionales del estado de emergencia actual no se pueden realizar las técnicas de investigación directa, ni las entrevistas, ya que en una entrevista, además de las respuestas de los usuarios, es importante analizar las acciones, reacciones y demás mediciones, que no se pueden realizar correctamente en una entrevista online.

  \section{Análisis de los datos}

  \subsection{Cuestionario}

  Antes de realizar el análisis, se ha decidido eliminar las respuestas que se consideraban inútiles para el análisis, como las respuestas donde se puede ver que las respuestas introducidas no son reales, además de eliminar todas las respuestas que respondieron negativamente a la última pregunta, ya que eso indica que la persona que respondió no es un usuario potencial de la aplicación.

  \subsubsection{Edad de los usuarios}

  \imgcenter[140]{Images/edad_usuarios.png}

  Se aprecia que el rango de edades abarca desde los 14 años hasta los 69. También se diferencian claramente dos grupos de edades con más respuestas útiles: Grupo alrededor de los 20 años, y grupo alrededor de los 50. Esto corresponde con los dos grupos de usuarios descritos, donde el primer grupo principalmente ofrecerá servicios, y el segundo ofrecerá y solicitará servicios.

  \vspace{4mm}




  \subsubsection{Frecuencia de uso de aplicaciones similares}

  \includegraphics[width=0.5\linewidth]{Images/frec_just.png}\hfill \hfill\includegraphics[width=0.5\linewidth]{Images/frec_wall.png}

  Se observa que hay un alto porcentaje de personas (55\%) que no han usado aplicaciones como \textit{Just Eat}, \textit{Uber Eats}, \textit{Glovo}... . En cambio, en torno a un 70\% de personas han usado aplicaciones como \textit{Wallapop} alguna vez, esto indica que podría haber un gran porcentaje de usuarios potenciales de la aplicación.

  \vspace{4mm}

  Además, el 55\% de personas que no han usado apps como Just Eat, Uber Eats, etc, puede ser porque estas \textit{apps} estan orientadas o las consumen un público más joven puesto que son de comida y pueden no interesarles a usuarios de más edad.

  \vspace{4mm}

  El hecho de que haya tantos usuarios potenciales con conocimientos sobre aplicaciones similares facilitará el uso de la aplicaicón, ya que ofrecerá un servicio con el que el usuario ya está familizarizado.


  \subsubsection{Aceptación de la aplicación}

  \imgcenter[140]{Images/utilizacion.png}

  Como se ve en la gráfica, hay un 68\% de personas que estarían dispuestos a usar la aplicación para solicitar trabajos puntuales. Además, hay un 25\% que podría no convencerles del todo el uso de esta aplicación. Sin embargo, se podría buscar la forma de convencer a este grupo de usuarios que, de primeras, no lo ven como una buena idea.

  \imgcenter[140]{Images/ofrecer.png}

  En esta gráfica se aprecia que alrededor del 78\% de las respuestas son positivas a la hora de ofrecer servicios, lo cual indica una buena aceptación por parte de los entrevistados.

  \vspace{4mm}


  Con estos datos puede concluirse que la aceptación es bastante buena ya que hay un gran número de personas a las que les parece una buena opción aunque hay otro grupo (25\%)a los que no les convence del todo, pero que podrían darle una oportunidad. Por lo tanto, se trata de crear una aplicación que llame la atención de todos los usuarios, tanto de los más experimentados en otras apps como de los que menos, con el objetivo de englobarlos y que utilicen esta aplicación.

  \section{Definición de Requisitos y Guías para el Diseño}

  \subsection{Requisitos}

  \subsubsection{Requisitos Funcionales}

  \begin{center}
    \begin{tabular}{|p{2cm}|p{2cm}|p{1cm}|p{6cm}|}
      \hline
      \textbf{ID} & RF-01 & \textbf{Título} & Inicio de Sesión \\
      \hline
      \textbf{Prioridad} & Alta & \textbf{Tipo} & Requisito Funcional de Usuario \\
      \hline
      \textbf{Descripción} & \multicolumn{3}{|p{9cm}|}{La aplicación deberá permitir a los usuarios registrados identificarse como usuarios.} \\
      \hline
    \end{tabular}

    \vspace{1cm}

    \begin{tabular}{|p{2cm}|p{2cm}|p{1cm}|p{6cm}|}
      \hline
      \textbf{ID} & RF-02 & \textbf{Título} & Creación de Cuenta \\
      \hline
      \textbf{Prioridad} & Alta & \textbf{Tipo} & Requisito Funcional de Usuario \\
      \hline
      \textbf{Descripción} & \multicolumn{3}{|p{9cm}|}{La aplicación deberá permitir a los usuarios no registrados crearse una cuenta.} \\
      \hline
    \end{tabular}

    \vspace{1cm}

    \begin{tabular}{|p{2cm}|p{2cm}|p{1cm}|p{6cm}|}
      \hline
      \textbf{ID} & RF-03 & \textbf{Título} & Acceso a la Cuenta \\
      \hline
      \textbf{Prioridad} & Alta & \textbf{Tipo} & Requisito Funcional de Usuario \\
      \hline
      \textbf{Descripción} & \multicolumn{3}{|p{9cm}|}{La aplicación deberá permitir a los usuarios registrados acceder a la información de su perfil.} \\
      \hline
    \end{tabular}

    \vspace{1cm}

    \begin{tabular}{|p{2cm}|p{2cm}|p{1cm}|p{6cm}|}
      \hline
      \textbf{ID} & RF-04 & \textbf{Título} & Actualización de Perfil \\
      \hline
      \textbf{Prioridad} & Alta & \textbf{Tipo} & Requisito Funcional de Usuario \\
      \hline
      \textbf{Descripción} & \multicolumn{3}{|p{9cm}|}{La aplicación deberá permitir a los usuarios registrados modificar la información de su perfil.} \\
      \hline
    \end{tabular}

    \vspace{1cm}

    \begin{tabular}{|p{2cm}|p{2cm}|p{1cm}|p{6cm}|}
      \hline
      \textbf{ID} & RF-05 & \textbf{Título} & Solicitud de Servicios \\
      \hline
      \textbf{Prioridad} & Alta & \textbf{Tipo} & Requisito Funcional de Usuario \\
      \hline
      \textbf{Descripción} & \multicolumn{3}{|p{9cm}|}{La aplicación deberá permitir a los usuarios registrados solicitar servicios de otros usuarios.} \\
      \hline
    \end{tabular}

    \vspace{1cm}

    \begin{tabular}{|p{2cm}|p{2cm}|p{1cm}|p{6cm}|}
      \hline
      \textbf{ID} & RF-06 & \textbf{Título} & Convertirse en \textit{H3lper} \\
      \hline
      \textbf{Prioridad} & Alta & \textbf{Tipo} & Requisito Funcional de Datos \\
      \hline
      \textbf{Descripción} & \multicolumn{3}{|p{9cm}|}{La aplicación deberá permitir a los usuarios registrados iddentificarse como \textit{H3lper}.} \\
      \hline
    \end{tabular}

    \vspace{1cm}

    \begin{tabular}{|p{2cm}|p{2cm}|p{1cm}|p{6cm}|}
      \hline
      \textbf{ID} & RF-07 & \textbf{Título} & Publicar Servicios \\
      \hline
      \textbf{Prioridad} & Alta & \textbf{Tipo} & Requisito Funcional de Usuario \\
      \hline
      \textbf{Descripción} & \multicolumn{3}{|p{9cm}|}{La aplicación deberá permitir a los \textit{H3lper} publicar servicios.} \\
      \hline
    \end{tabular}

  \end{center}



  \vspace{1.5cm}

  \subsection{Requisitos No Funcionales}

  \begin{center}
    \begin{tabular}{|p{2cm}|p{2cm}|p{1cm}|p{6cm}|}
      \hline
      \textbf{ID} & RNF-01 & \textbf{Título} & Plataforma \\
      \hline
      \textbf{Prioridad} & Alta & \textbf{Tipo} & Requisito No Funcional Ambiental \\
      \hline
      \textbf{Descripción} & \multicolumn{3}{|p{9cm}|}{La aplicación deberá poder ser \textit{responsive} para ser visualizada en cualquier tipo de pantalla.} \\
      \hline
    \end{tabular}

    \vspace{1cm}

    \begin{tabular}{|p{2cm}|p{2cm}|p{1cm}|p{6cm}|}
      \hline
      \textbf{ID} & RNF-02 & \textbf{Título} & Navegadores \\
      \hline
      \textbf{Prioridad} & Alta & \textbf{Tipo} & Requisito No Funcional Ambiental \\
      \hline
      \textbf{Descripción} & \multicolumn{3}{|p{9cm}|}{La aplicación deberá poder ejecutarse en los siguientes navegadores web: \textit{Mozilla Firefox}, \textit{Google Chrome} y \textit{Safari}.} \\
      \hline
    \end{tabular}

    \vspace{1cm}

    \begin{tabular}{|p{2cm}|p{2cm}|p{1cm}|p{6cm}|}
      \hline
      \textbf{ID} & RNF-03 & \textbf{Título} & Multidispositivo \\
      \hline
      \textbf{Prioridad} & Alta & \textbf{Tipo} & Requisito No Funcional de Datos \\
      \hline
      \textbf{Descripción} & \multicolumn{3}{|p{9cm}|}{La cuenta de los usuarios podrá ser accedida desde distintos dispositivos simultaneamente.} \\
      \hline
    \end{tabular}
  \end{center}

  \vspace{1.5cm}

  \subsection{Glosario}

  \begin{itemize}
    \item \textbf{\textit{H3lper}}: vocabulario técnico de la aplicación. Usuario que puede ofrecer y publicar servicios en la aplicación.
    \item \textbf{Identificarze como usuarios}: introducir los datos de usuario y contraseña validos para iniciar sesión en la aplicación.
    \item \textbf{Perfil}: página de la aplicación personalizada específicamente para un usuario, y que recoge toda su información.
    \item \textbf{Publicar un servicio}: hacer visible a todos los usuarios un servicio concreto.  
    \item \textbf{\textit{Responsive}}: técnica de diseño web que busca la correcta visualización de una misma página en distintos dispositivos.
    \item \textbf{Servicios}: En el contexto de esta aplicación, se considera servicio a toda acción o trabajo especializado ofrecido por los usuarios.
  \end{itemize}



\end{document}
