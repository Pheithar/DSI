\documentclass{uc3mpracticas}



%%%%%%%%%%%%%%%%%%%%%%%%%%%%%%%%%%%%%%%%%%%%%%%%%%%%%%%%%%%%%%%%%%%%%%%%%%%%%%%%
%%%                   Plantilla Prácticas UC3M                               %%%
%%%                Universidad Carlos III de Madrid                          %%%
%%%                   Alejandro Valverde Mahou                               %%%
%%%%%%%%%%%%%%%%%%%%%%%%%%%%%%%%%%%%%%%%%%%%%%%%%%%%%%%%%%%%%%%%%%%%%%%%%%%%%%%%

%Permitir cabeceras y pie de páginas personalizados
\pagestyle{fancy}

%Path por defecto de las imágenes
\graphicspath{ {./images/} }

%Declarar formato de encabezado y pie de página de las páginas del documento
\fancypagestyle{doc}{
  %Cabecera
  \headerpr[1]{Análisis y Estudio de Campo}{\textbf{H3lpMe}}{Diseño de Sistemas Interactivos}
  %Pie de Página
  \footerpr{}{}{{\thepage} de \pageref{LastPage}}
}

%Declarar formato de encabezado y pie del título e indice
\fancypagestyle{titu}{%
  %Cabecera
  \headerpr{}{}{}
  %Pie de Página
  \footerpr{}{}{}
}


\appto\frontmatter{\pagestyle{titu}}
\appto\mainmatter{\pagestyle{doc}}


\begin{document}
  %Comienzo formato título
  \frontmatter


  %Portada 1 (Centrado todo)
  \centeredtitle{Images/LogoUC3M.png}{Grado en Ingeniería Informática}{Curso 2020}{Diseño de Sistemas Interactivos}{Análisis y Estudio de Campo: H3lpMe}

  \vspace{40mm}

  \begin{center}
    \line(1, 0){450}
  \end{center}

  \authorsright{Alejandro Parrado Rivas}{100383453}{Adrián Sanz Gómez}{100383473}{Alejandro Valverde Mahou}{100383383}{Andrés Vinagre Blanco}{100383414}

  \newpage


  %Índice
  \tableofcontents

  \newpage

  %Comienzo formato documento general
  \mainmatter

  \section{Introducción}

  \textbf{H3lpMe} es una aplicación web a través la cual los usuarios pueden pedir ayuda a otros en cualquier ámbito de la vida cotidiana o profesional, como puede ser un servicio de comida, resolver un problema informático, o cualquier incidente que pueda surgir en el día a día.

  \vspace{4mm}

  El objetivo fundamental de la aplicación es hacer más fácil el día a día de las personas, ayudando en aquellas actividades en las que los usuarios no pueden desenvolverse, sin tener la necesidad de contratar a un profesional o realizar trámites complejos a través de intermediarios.

  \vspace{4mm}

  En la app, cualquier persona puede crearse un perfil indicando sus habilidades y capacidades, para que otros usuarios de la plataforma puedan solicitarlas cuando lo necesiten acordando un precio entre ellos.

  \section{Contexto de Diseño}

  En un estudio previo, se pueden definir dos grupos de \textit{usuarios} principales:

  \vspace{4mm}

  El primer grupo está formado por personas con una edad comprendida entre los 25 y los 50 años. Este grupo de usuarios ofrecerá y solicitará servicios que tengan que ver con el cuidado y mantenimiento de un hogar, como puede ser servicio de reparaciones, limpieza o cocina. Su conocimiento de los sistemas informáticos y aplicaciones no es experto, pero los conocen y se pueden manejar con facilidad con ellos.

  \vspace{4mm}

  El segundo grupo está formado po personas con una edad comprendida entre los 16 y los 25 años. El propósito que le dan a estos usuarios a la aplicación estará más centrado a los estudios, por lo que los servicios que ofrecerán o solicitarán será principalmente el de las clases particulares, pudiendo también ofrecer servicios de cuidado de niños pequeños u potras tareas, pero normalmente no solicitándolas.

  % ESTO NECESITA UNA REVISIÓN Y SEGUNDA REDACCIÓN


  \vspace{8mm}

  Como \textit{stakeholders} se pueden definir varios grupos:

  \begin{itemize}
    \item \textbf{Usuarios finales de la aplicación}, que ya hemos dividido previamente en dos grupos.
    \item \textbf{Programadores y desarrolladores}, que son las personas ivolucradas en el desarrollo de ka aplicación. Deben conocer el funcionamiento de la aplicación completa, y son los encargados de que se cumplan las espectativas de los usuarios.
    \item \textbf{Diseñadores}, que necesitan realizar prototipos y diseños que cuplan las necesidades de los usuarios, y sean intuitivos y completos.
  \end{itemize}


  \vspace{8mm}





  \section{Diseño y Desarrollo de Técnicas de Investigación del Trabajo de Campo: Obtención de Datos}

  Para la obtención de datos se ha elegido usar cuestionarios online a través de los \textit{Formularios de Google}. Esto permite una gran difusión en poco tiempo. Además, se realizará una investigación sobre el estado actual del arte, las apliuacaciones actuales que ofrecen un servicio similar, y las posibles leyes que habrá que tener en cuenta a la hora de ofrecer los servicios de la aplicación al público.

  \subsection{Cuestionario}

  El cuestionario posee las siguientes preguntas para los potenciales usuarios:

  \begin{itemize}
    \item \textbf{¿Cuántos años tienes?}:
      Esta pregunta permite conocer el rango de edad de los posibles usuarios

    \item \textbf{¿Has usado alguna vez Wallapop o aplicaciones similares de compra/venta?}:
      Esta pregunta permite conocer la familiaridad que tienen los posibles usuarios con aplicaciones que ofrecen un servicio parecido, ya que, al igual que la aplicación a desarrollar, se basan en la confianza entre desconocidos para, en el caso de \textit{Wallapop}, la compra-venta de productos, y en el caso de la aplicación a desarrollar, el ofrecimiento de servicios.

    \item \textbf{¿Has usado alguna vez aplicaciones como Just Eat, Uber Eats o similares de comida a domicilio?}:
      Esta pregunta permite conocer la familiaridad que tienen los posibles usuarios con aplicaciones que ofrecen servicios a domicilio, ya que la aplicación a desarrollar también posee este servicio.

    \item \textbf{¿Alguna vez has necesitado un profesor particular/fontanero/técnico informático/pintor o albañil/niñero o cuidador?}:
      Esta pregunta permite conocer las necesidades de los posibles usuarios (cada '/' supone una pregunta diferente, con el mismo formato).

    \item \textbf{En caso afirmativo, ¿cómo le conociste?}:
      Esta pregunta permite conocer como los poibles usuarios suelen buscar actualmente gente que ofrezca los servicios que necesitan.

    \item \textbf{¿Cuáles de los siguientes servicios te ves capacitado a ofrecer?}:
      Esta pregunta facilita la identificación de posibles usuarios que ofrecerán servicios dentro de la aplicación, y cuales son los servicios que podrán ofrecer (las opciones facilitadas son : \textit{Profesor Particular}, \textit{Fontanero}, \textit{Técnico Informático}, \textit{Pintor}, \textit{Niñero}, \textit{Cuidador}, \textit{Albañil}, \textit{Cocinero}, \textit{Limpieza del hogar}, \textit{Mudanzas}, \textit{Fotógrafo}).

    \item \textbf{¿Confiarías en alguien que no conoces para realizar alguno de los servicios mencionados anteriormente?}:
      Esta pregunta permite conocer si los posibles usuarios que requieran de servicios aceptarían a otros usuarios que ofrecieran esos servicios en sus hogares.

    \item \textbf{¿Si buscaras trabajo en alguna de las áreas anteriores o similares, te registrarías en una aplicación donde soliciten trabajos puntuales a domicilio?}:
      Esta pregunta permite identificar posibles usuarios que ofrecerían servicios a través de la aplicación.

    \item \textbf{¿Si necesitaras solicitar un trabajo puntual en alguna de las áreas anteriores o similares, te registrarías en una aplicación donde personas de diferente experiencia ofrezcan sus servicios?}:
      Esta pregunta permite identificar a los posibles usuarios que solicitarían servicios a través de la aplicación.

  \end{itemize}


  % REVISAR
  Además, por cuestiones excepcionales del estado de emergencia actual no se pueden realizar las técnicas de investigación directa, ni las entrevistas, ya que en una entrevista, además de las respuestas de los usuarios, es importante analiar las acciones, reacciones y demás mediciones, que no se pueden realizar correctamente en una entrevista online.

  \section{Análisis de los datos}

  \section{Definición de Requisitos y Guías para el Diseño}






\end{document}
