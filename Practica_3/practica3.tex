\documentclass{uc3mpracticas}


\usepackage{float}

%%%%%%%%%%%%%%%%%%%%%%%%%%%%%%%%%%%%%%%%%%%%%%%%%%%%%%%%%%%%%%%%%%%%%%%%%%%%%%%%
%%%                   Plantilla Prácticas UC3M                               %%%
%%%                Universidad Carlos III de Madrid                          %%%
%%%                   Alejandro Valverde Mahou                               %%%
%%%%%%%%%%%%%%%%%%%%%%%%%%%%%%%%%%%%%%%%%%%%%%%%%%%%%%%%%%%%%%%%%%%%%%%%%%%%%%%%

%Permitir cabeceras y pie de páginas personalizados
\pagestyle{fancy}

%Path por defecto de las imágenes
\graphicspath{ {./images/} }

%Declarar formato de encabezado y pie de página de las páginas del documento
\fancypagestyle{doc}{
  %Cabecera
  \headerpr[1]{Fase de Análisis y Evaluación}{\textbf{}}{Diseño de Sistemas Interactivos}
  %Pie de Página
  \footerpr{\textbf{Universidad Carlos III de Madrid}}{}{{\thepage} de \pageref{LastPage}}
}

%Declarar formato de encabezado y pie del título e indice
\fancypagestyle{titu}{%
  %Cabecera
  \headerpr{}{}{}
  %Pie de Página
  \footerpr{}{}{}
}


\appto\frontmatter{\pagestyle{titu}}
\appto\mainmatter{\pagestyle{doc}}


\begin{document}
  %Comienzo formato título
  \frontmatter


  \centeredtitle{Images/logo_help.png}{Grado en Ingeniería Informática}{Curso 2020}{Diseño de Sistemas Interactivos}{Fase de Análisis y Evaluación: H3lp Me}

    \vspace{2mm}

    \begin{center}
      \line(1, 0){450}
    \end{center}

    \authorsright{Alejandro Parrado Arribas}{100383453}{Adrián Sanz Gómez}{100383473}{Alejandro Valverde Mahou}{100383383}{Andrés Vinagre Blanco}{100383414}

    \newpage


    %Índice
    \tableofcontents

    \listoffigures

  \newpage

  %Comienzo formato documento general
  \mainmatter

  \section{Introducción}

  La fase de análisis y evaluación es necesaria en todos los proyectos de desarrollo de software, para conocer los fallos en funcionalidad o diseño que se hayan producido en los mismos y, por tanto, poder arreglarlos, de forma que el producto está en continuo cambio y crecimiento, suponiendo una mejora.

  \vspace{3mm}

  Para conocer estos posibles fallos, será necesario saber la opinión de usuarios finales y de profesionales de forma que cada uno aporte su punto de vista con respecto a la aplicación.

  \vspace{4mm}

  Los usuarios finales aportarán información desde el punto de vista práctico. Podrán informar  qué es intuitivo y qué no, qué es lo que esperaban de la aplicación  y lo que han conseguido. Es importante recoger información de usuarios principiantes y de usuarios expertos puesto que cada uno tienen una perspectiva diferente.

  \vspace{2mm}

  Los expertos aportarán información más técnica pero igual de importante sobre el diseño, implementación y demás aspectos de la aplicación que los usuarios suelen pasar por alto pero que afectan en gran medida al uso de la aplicación.

  \vspace{7mm}

  La aplicación a analizar y evaluar es la desarrollada en prácticas anteriores : \textit{H3lp Me}, accesible desde el link \url{https://h3lp-me.web.app}.

  \vspace{3mm}

  La aplicación permite a usuarios que quieran ofrecer servicios puntuales, como pueden ser cocina, cuidado de niños o fotografía, ponerse a contacto con aquellos que buscan estos servicios.

  \vspace{2mm}

  La aplicación consiste en tres páginas principales:

  \vspace{2mm}
  \begin{center}
    \begin{figure}[H]
      \caption{\textit{Página Principal}}
      \vspace{2mm}
      \imgcenter[160]{Images/main_page.png}
    \end{figure}

    \begin{figure}[H]
      \caption{\textit{Página de Servicios}}
      \vspace{2mm}
      \imgcenter[160]{Images/services_page.png}
    \end{figure}

    \begin{figure}[H]
      \caption{\textit{Página de Perfil}}
      \vspace{2mm}
      \imgcenter[160]{Images/profile_page.png}
    \end{figure}
  \end{center}

  Cada una de estas tres pantallas cumple un propósito concreto: La página principal da la bienvenida a nuevos usuarios, informa del objetivo de la web  y  de las distintas acciones que puede hacer un usuario. La página de servicios muestra los servicios que pueden ser contratados en la página. La página de perfil permite personalizar el perfil, visualizar los servicios contratados, y publicar nuevos servicios.


  \section{Técnicas de Evaluación}

  Para la evaluación se usarán dos ténicas diferentes, cada una con un propósito diferente.

  \vspace{4mm}

  \textbf{Verificación de requisitos}: Esto permitirá ver qué se ha quedado en el tintero y no ha podido ser realizado, a pesar de estar definido como requisito de la práctica.

  \vspace{2mm}

  \textbf{Estudios de usuarios}: Gracias a entrevistas \textit{online} a usuarios se podrá obtener la información necesaria de los usuarios para conocer la acepactión de los mismo hacia la aplicación.


  \subsection{Verificación de Requisitos}

  \subsubsection{Protocolo}

  Esta evaluación tiene como propósito verificar qué propósitos se han cumplido parcialmente o no se han cumplido para que en futuras actualizaciones de la página se solucionen.

  \vspace{2mm}

  Se llevará a cabo por los creadores de la aplicación ya que no requiere de ningún usuario ni experto.

  \vspace{2mm}

  Esta verificación, como no puede ser de otra forma, se hará revisando los requisitos definidos en la \textbf{fase de análisis y estudio de campo}.

  \vspace{2mm}

  Los datos que se obtendrán de esta técnica serán de tipo cualitativos e informarán de los puntos a tener en cuenta en las siguientes actualizaciones. También informarán de la completitud de la aplicación respecto a la previsión de la misma.

  \subsubsection{Evaluación}

  \textbf{Requisitos No Cumplidos}

  \begin{itemize}
    \item \textbf{RF-06: Convertirse en H3lper}

    La aplicación no tiene forma de, una vez el usuario está registrado, cambiar su estado de no \textit{H3lper} a \textit{H3lper}.

    \item \textbf{RF-08: Valoracion H3lper}

    La aplicación no tiene forma de valorar a los \textit{h3lper} una vez el servicio ha sido realizado.

    \item \textbf{RF-12: Comunicación entre Usuarios}

    La aplicación no tiene forma de comunicar distintos usuarios de ninguna manera.

    \item \textbf{RNF-01: Plataforma}

    La aplicación no es responsive. Solo se visualiza correctamente en ordenadores.

  \end{itemize}

  \textbf{Requisitos Cumplidos Parcialmente}

  \begin{itemize}
    \item \textbf{RNF-02: Navegadores}

    La aplicación muestra un estilo diferente en el navegador \textit{Safari}.

    \item \textbf{RNF-04: Creación de cuentas}

    No existe campo de \textit{email} en la creación de usuarios.
  \end{itemize}

  \subsection{Estudios de usuarios}

  \subsubsection{Protocolo}

  El objetivo de esta técnica es conocer el punto de vista de los usuarios, para poder saber los fallos a nivel usuario que existen en la aplicación. Esta técnica es muy útil y muy usada. Para que la técnica sea completa, se tiene que evaluar a varios usuarios.

  \vspace{2mm}

  Los usuarios serán entrevistados en persona si es posible, y si no, serán entrevistados \textit{online} a través de comunciación por voz con los evaluadores, mientras comparten panatlla para que los evaluadores puedan ver sus acciones.

  \vspace{2mm}

  Las preguntas que se harán serán las siguientes:

  \begin{enumerate}
    \item \textbf{¿De que piensas que va la página web?}

    Esta pregunta se hará a los usuarios nada más acceder a la página y haber visto la página principal. Esto permitirá evaluar si la página principal expresa de forma correcta la intención de la página.

    \item \textbf{Create una cuenta con categoría \textit{h3lper} y crea un servicio con categoría otros y con la imágen por defecto. Después vuelve a la página principal}

    Para este apartado será necesario medir el tiempo que tarda el usuario en realizar esta tarea y el número de veces que se equivoca. De esta forma se podrá ver si la página tiene una interfaz sencilla y entendible para los usuarios nuevos.

    \item \textbf{Busca un servicio llamado 'Servicio de Prueba' con categoría 'Informática' y solicitalo.}

    Este apartado será tratado como el anterior, pues prueba la segunda funcionalidad de la aplicación.

    \item \textbf{Cambia tu imagen de perfil}

    Esto prueba una funcionalidad del perfil de usuario que quizá no queda claro como hacer a primera vista, por lo que es interesante saber si el usuario es capaz de hacerlo sin ayuda externa. Se deben anotar los fallos que cometan los usuarios.

    \item \textbf{Cierra sesión}

    Permite verificar que el usuario es capaz de encontrar el botón de cerrar sesión correctamente. Se deben anotar los fallos que cometan los usuarios.

    \begin{center}
      \line(1, 0){450}
    \end{center}

    \textbf{\textit{En este momento hay que otorgarle manualmente desde la base de datos monedas extra al usuario, antes de pasar a la siguiente pregunta}}

    \begin{center}
      \line(1, 0){450}
    \end{center}

    \item \textbf{Inicia Sesión y promociona el sevicio que habías creado anteriormnente para que se mantenga promocionado durante 3 días}

    Esto permitira verificar otra funcionalidad de la aplicación, y al igual que en el resto de casos, será necesario apuntar el tiempo que tarda y el núemro de fallos que comete.

    \begin{center}
      \line(1, 0){450}
    \end{center}

    \textbf{\textit{Aquí terminan las preguntas técnicas y comienzan las subjetivas. Se le tiene que pedir al usuario que cierre la página}}

    \begin{center}
      \line(1, 0){450}
    \end{center}

    \item \textbf{En tu opinión, ¿qué le hace falta a la página?}

    Esto permite a los usuarios expresar su opinión sobre que necesita la página para ser más completa.

    \item \textbf{En tu opinión, ¿qué es lo que menos te ha gustado de la página?}

    Esto permite expresar a los usuarios sus opiniones sobre la página. Es importate informarle de que puede expresarse sin miedo.

    \item \textbf{En tu opinión, ¿qué es lo que más te ha gustado de la página?}

    Permite resaltar los puntos fuertes de la página, para poder ser potenciados.

    \item \textbf{¿Recuerdas el nombre de la página?}

    Permite comprobar si se ha creado una senscaión de marca correctamente.



  \end{enumerate}

  Los miembros del equipo cumplirán las funciones de evaluadores y deberán apoyar a los usaurios pero sin dar pistas ni informar  de cómo deben realizar la tarea ya que el diseño debe ser suficicentemente intuitivo como para que los usuarios sepan hacerlo. En caso de no saber hacer algo, simplemente se pasará a la siguiente pregunta, apuntando que el diseño ha fallado en ese caso. No se debe presionar a los usuarios para que completen las tareas lo más rápidamente posible ni juzgar sus fallos.

  \vspace{4mm}

  Los datos obtenidos de esta prueba son de suma importancia para saber si el diseño es correcto o no y para conocer las perspectivas de los usuarios.

  \vspace{4mm}

  Los datos obtenidos tras realizar entrevistas a 10 usuarios distinto, con distinto nivel de conocimiento y habilidad con páginas similares so:

  \begin{enumerate}
    \item \textbf{Pregunta 1}:

    La mayoría de usuarios puede discernir la temática en general, pero no suelen ser capaces de decir con exactitud lo que esperan conseguir dentro de la página.

    \item \textbf{Pregunta 2}

    El tiempo medio de los usuarios para conseguir realizar la tarea propuesta en la pregunta 2 es de \textbf{2 minutos y 2 segundos}.

    \vspace{2mm}

    Algunos usuarios han cometido fallos en esta tarea pero no parece haber un error recurrente.

    \item \textbf{Pregunta 3}

    El tiempo medio de los usuarios para conseguir realizar la tarea propuesta en la pregunta 3 es de \textbf{22 segundos}.

    \vspace{2mm}

    Un error recurrente en esta tarea es que los usuarios en la barra de navegación intentan pulsar \texttt{Enter}, que no está vinculada, y no entienden por que no funciona  y no diferencia correctamente el botón de \textit{Buscar}.

    \vspace{1mm}

    Además, muchas veces intentan entrar en el \textit{Servicio} pulsando en la imagen o el título, en lugar de pulsando el botón de \textit{+ INFO}.


    \item \textbf{Pregunta 4}

    Una gran proporción de los usuarios no encontraban cómo cambiar su imagen de perfil. El \textbf{80\%} intento cambiarlo en el menú de la esquina superior derecha.

    \item \textbf{Pregunta 5}

    En este apartado ningún usuario ha tenido problemas.

    \item \textbf{Pregunta 6}

    El tiempo medio de los usuarios para conseguir realizar la tarea propuesta en la pregunta 6 es de \textbf{1 minuto y 4 segundos}.

    \vspace{2mm}

    El fallo más recurrente en esta tarea es que intentaban promocionar el servicio dentro del mismo porque no encontraban el menú de la tienda dentro de su perfil.

    \item \textbf{Pregunta 7}

    Lo que los usuarios han considerado que falta más en la página es un diseño más estruturado y profesional. Además se quejan de que las imágenes que suben los usuarios son alteradas y mal distribuidas en la página. Otra solicitud recurrente es el añadir una guía de uso dentro de la propia página.

    \item \textbf{Pregunta 8}

    Lo que menos les ha gustado a los usuarios es su atractivo visual y la distribución de los apartados. Preferirían una división entre \textit{Servicios Solicitados},  \textit{Servicios Ofrecidos} y \textit{Tienda} para no tener que hacer \textit{scroll} en la misma página.

    \item \textbf{Pregunta 9}

    Prácticamente todos los usuarios están de acuerdo en que la página es muy fácil de usar y muy intuitiva. Además, algunos usuarios, en contraposición a la respuesta anterior, dicen que es bonita y que tiene un buen diseño estético.

    \item \textbf{Pregunta 10}

    Todos los usuarios han sido capaces de recordar correctamente el nombre de la página: \textit{H3lp Me}.

  \end{enumerate}


  \section{Análisis de Datos}

  \subsection{Análisis cuantitativo}

  El análisis cuantitativo se va a basar en la obtención y recopilación de datos mediante gráficas.

  \vspace{2mm}
  \imgcenter[160]{Images/data3.jpg}

  La primera gráfica se corresponde al tiempo que tardan los usuarios entrevistados en realizar las tareas propuestas. Se  observa que la primera tarea; correspondiente a la pregunta 2, ha sido la que más ha tardado la gente ya que debian de crearse una cuenta de usuario y era su primer contacto con la aplicación.
  Respecto a a la tercera pregunta, los entrevistados tardaron notablemente menos que la anterior porque tenían que solicitar un servicio y se nota que ya se familiarizaron un poco más con la interfaz. La pregunta número seis se pedía promocionar un servicio, esta tarea requiere un poco más de tiempo que la anterior porque es necesario obtener previamente las monedas suficientes para promocionarlo.

  \vspace{2mm}
  \imgcenter[160]{Images/data2.jpg}
  Respecto a las opiniones obtenidas, la mayoría de ellas son buenas. Destacamos que la aplicación es fácil, bonita e intuitiva; precisamente uno de los objetivos que se buscaban.

  \vspace{2mm}
  \imgcenter[160]{Images/data1.jpg}
  Para conocer el efecto e importancia que tiene nuestra marca en los usuarios, destacamos que el 80 porciento de los entrevistados se acuerda exactamente del nombre de la aplicación, recalcando que a varios de ellos les ha gustado el logo.

  \subsection{Análisis cualitativo}

  Para analizar los datos cualitativos obtenidos en la fase de Técnicas de Evaluación vamos a utilizar la \textbf{Técnica del Incidente Crítico}. Esta técnica se define como un conjunto de procedimientos utilizados para la recolección de observaciones directas de la conducta humana que tienen una importancia crítica y que cumplen con definir criterios metódicamente.

  Esta técnica se divide en cinco áreas: determinar y revisar el incidente, determinar los hechos, identificar los problemas, resolver los problemas y por último la evaluación de dicha solución.

  En nuestro caso las dos primeras áreas se completan con las fases anteriores, donde a través de una evaluación de requisitos y de las preguntas a usuarios hemos determinado los incidentes y recolectado los hechos (datos). A continuación pasaremos a analizar la 3 áreas siguientes para cada uno de los incidentes encontrados.

  \subsubsection{Evalución de Requisitos}

  Para la parte de \textbf{Evaluación de Requisitos} la identificación de los problemas también se llevó a cabo en la fase anterior así que pasaremos a las dos últimas fases directamente:

  \begin{enumerate}

    \item \textbf{RF-06: Convertirse en H3lper}
    Para solucionar el problema de poder cambiar el estado de no \textit{H3lper} a \textit{H3lper} una vez se ha creado la cuenta se podría implementar un botón dentro de los ajustes del perfil del usuario, de tal forma que trás confirmar que si un usuario quiere convertirse en helper le sean visibles todas las opciones y menús que conlleva. Un ejemplo de esto podría ser las opciones de desarrollador de los smartphones, que una vez se activan te dejan acceder a menús que antes no eran visibles para el usuario.

    \vspace{2mm}

    Para comprobar si la solución ha solucionado el problema, valdría con crear una cuenta de prueba que no fuera H3lper y probar el botón añadido, para ver si a partir de ahí puede acceder a las características de los H3lpers.

    \item \textbf{RF-08: Valoracion H3lper}
    El problema era que la aplicación no daba opción a valorar al H3lper una vez el servicio había sido acabado. Una solución posible dada esta descripción sería introducir un pequeño cuestionario de valoración una vez se termina el servicio, como por ejemplo usan las aplicaciones de llamadas online al terminar la conversación.

    \vspace{2mm}

    Para comprobar si está solución es válida valdría con solicitar un servicio y ver si el cuestionario aparece una vez termine y si la valoración se añade a la actual del usuario que ofrece dicho servicio.

    \item \textbf{RF-12: Comunicación entre Usuarios}
    El problema en este requisito era que no existía método de comunicación entre usuarios. En el prototipo final de la aplicación existía un apartado para iniciar y continuar chats con los usuarios vinculados a los servicios, sin embargo era muy complicado de implementar y llevaba demasiado tiempo. La solución pues, sería terminar esa implementación utilizando \textbf{Firebase} para almacenar los datos de las conversaciones.

    \vspace{2mm}

    En la comprobación necesitaríamos dos usuarios de prueba, uno de ellos ofrecería un servicio y el otro lo contrataría. Una vez hecho esto se pemite iniciar un chat entre ambos y ahí se podría comprobar si los mensajes se están almacenando correctamente en la base de datos.

    \item \textbf{RNF-01: Plataforma}
    En este requisito no funcional el problema era que la aplicación no era responsive, es decir, no se ajustaba a todos los dispositivos. Dado que centramos el desarrollo a una interfaz desde PC, muchas de las ventanas e interfaces no se ven correctamente en dispositivos móviles. Esto se podría solucionar modificando algunos tamaños de elemntos de la página, para que todo fuera en porcentajes o también se podría añadir alguna cabecera predefinida para ajustar las dimensiones a otros dispositivos como tienen muchas páginas web actuales.

    \vspace{2mm}

    Para comprobar que este problema se ha solucionado deberemos entrar en la app desde diferentes dispositivos electrónicos y acceder a las diferentes funciones de la misma, de tal forma que se pueda corroborar que todas las interfaces han sido actualizadas correctamente.

  \end{enumerate}

    \subsubsection{Estudio de Usuarios}

    En esta parte vamos a analizar los resultados obtenidos de las preguntas realizadas con valor cualitativo, es decir, aquellas que proveen información subjetiva acerca de la página. A partir de las respuestas de los usuario identificaremos los incidentes críticos y estudiaremos posibles mejoras.

    \begin{enumerate}

      \item \textbf{Pregunta 1}
      Esta pregunta estaba enfocada a conocer si a primera vista se entiende el tema de la página y como funciona. Los resultados indicaron que la mayoría de los usuarios tenían la temática clara pero encontramos una incidencia en el objetivo de la misma, es decir, el usuario no tiene claro para qué sirve la página. Este puede ser un problema mayor de lo que parece, ya que puede llevar a confusión y posterior pérdida de usuarios al entrar a la página. Una solución posible a este problema sería añadir una pequeña introducción en forma de pop-up para los nuevos usuarios (que no tengan cookies previas) donde se explique qué puede obtener en la página.

      \item \textbf{Pregunta 7}
      Esta pregunta intenta destapar vulnerabilidades y fallos de diseño en la página. Gracias a las opiniones de los usuarios logramos indicar 3 incidencias críticas que pueden llevar a problemas en la web.
      La primera de ellas es la falta de estructura profesional, este problema puede llevar a que algunos usuarios no se tomen en serio la página o no crea que sea lo suficicentemente útil. Para abarcar este problema se necesitaría una reestructuración estética de la página, si bien es cierto que el propio diseño fue escogido deliberadamente de esta forma para darle un aspecto más animado, puede ser positivo orientarlo a un área más profesional, para ello se necesitaría cambiar el formato de las cajas flex, el tipo de letra y los colores.

      \vspace{2mm}

      La segunda incidencia encontrada es relacionada a la subida de imágenes a la app, ya que en ciertas ocasiones aparecen distorsionadas en cuanto a la dimensión. Esto es debido a que para lograr un estilo uniforme se han encapsulado algunos espacios de imágenes con dimensiones fijas, de tal forma que al subirlas si no son proporcionales se ajustan. Una posible solución a esta incidencia sería obligar a subir fotos con determinadas dimensiones como hacen muchas otras páginas para, por ejemplo, las fotos de perfil de los usuarios.

      \vspace{2mm}

      Por último también encontramos la necesidad de los usuarios de añadir una guía de uso. Este problema realmente ya está solucionado aunque no del todo implementado. Con esto nos referimos a que en la página inicial se encuentra un "vídeo" que sería introductorio a como utilizar las funciones básicas de la página, aunque debido a la dificultad de crear un vídeo e introducirlo en la página decidimos se decidió dejarlo indicado.

      \item \textbf{Pregunta 8}
      El objetivo de esta pregunta era conocer los puntos débiles de la página, que saldrían destacados en caso de que varios usuarios tuvieran la misma opinión de qué es lo peor de la página. En nuestro caso identificamos una incidencia en la estructura del perfil. El problema reside en la necesidad de hacer scroll y en la estética general de la pestaña. Se podría solucionar poniendo los servicios ofrecidos y solicitados en diferentes ventanas, teniendo que clickar en ellos para poder acceder, de esta forma quedaría más organizado y dejaría el perfil más simple.

    \end{enumerate}

    Para poder comprobar si los cambios propuestos funcionan correctamente sería necesario realizar las preguntas nuevamente tras aplicar los cambios.

    \vspace{2mm}

    Con esto termina el análisis de las incidencias que afectan negativamente a la aplicación. Por contra, el resto de aspectos de la web han sido puntuados positivamente, en especial los usuarios han valorado la facilidad de uso y la estética.




  \section{Conclusión}

Respecto a las conclusiones generales de este análisis realizado hay que señalar que la página ha causado de forma general una buena impresión a los usuarios  ya que tiene un diseño simple y efectivo que, sin embargo, debe mejorarse haciéndose  más profesional, así como cambiar la distribución del perfil con el objetivo de tener la información mejor distribuida en el apartado del menú, servicios ofrecidos, solicitados y tienda. Asimismo, a pesar de que la intención de la página es clara, los usuarios en su primera toma de contacto no saben muy bien qué puede hacerse y qué no en la página, lo que puede solucionarse de forma fácil incluyendo una guía de uso al entrar por primera vez a la aplicación. A pesar de estos puntos negativos, a los usuarios les ha gustado el logo  la estética de la web y han recordado el nombre de la página.
Como reflexión crítica del diseño final  hay que señalar la intuitividad del mismo, sin embargo, es recomendable que se busque darle un diseño más profesional a la página así como hacerla responsive para poder adaptarse a cualquier tipo de dispositivo, lo que proporciona versatilidad y  aumenta la usabilidad de la misma. En la experiencia de usuario en la página, hay que implementar la  comunicación entre usuarios  mediante chats y la inclusión de una encuesta de valoración tras concluir un servicio, mejorar la subida de imágenes a la plataforma para que no se deformen (esto reduce su profesionalidad)  e implementar que un usuario pueda convertirse en H3lper una vez registrado en la web.
Por último, las metodologías de evaluación utilizadas están centradas en obtener los problemas de la página y  trazar aquellos conflictos que no permitan al usuario completar las tareas propuestas. De esta manera, se obtendrán los fallos y la solución es iterar sobre el diseño de la web hasta conseguir que estos desaparezcan y el usuario complete las tareas propuestas. Por lo que la técnica del \textbf{Técnica del Incidente Crítico} mencionada anteriormente nos ayuda a encontrar y proponer soluciones a los problemas de los usuarios al interactuar con la aplicación web.







\end{document}
